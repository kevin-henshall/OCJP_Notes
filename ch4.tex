\section{Chapter 4 -- Operators}
\subsection{Java Operators}
Java operators typically are not overloaded. There are however, a few 
exceptions:
\begin{itemize}
    \item The \verb#+# operator can be used to add two numeric primitives 
    together, or to perform a concatenation operation if either operand is a 
    String
    \item The \verb#&#, \verb#|# and \verb#^# operators can all be used for bit 
    manipulation and also for boolean conditionals
\end{itemize}

\subsection{Assignment Operators}
A summary of the assignment operator:
\begin{itemize}
    \item When assigning a value to a primitive, size matters
    \item Remember that a reference variable isn't an object -- it's a way to 
    get to an object
    \item When assigning a value to a reference variable, type matters.  
    Remember the rules for supertypes, subtypes and arrays
\end{itemize}

\subsection{Compound Assignment Operators}
The four most commonly used compound operators (\verb#+=#, \verb#-=#, 
\verb#*=#, \verb#/=#) are on the exam

\subsection{Relational Operators}
The exam covers six relational operators (\verb#<#, \verb#<=#, \verb#>#, 
\verb#>=#, \verb#==#, \verb#!=#) which always result in a \verb#boolean#. Java 
has four relational operators that can be used to compare \emph{any 
combination} of integers, floating-point numbers or characters (\verb#<#, 
\verb#<=#, \verb#>#, \verb#>=#). Note that when comparing a char with another 
char, or a char with a number, java will use the Unicode value of the char as 
the numerical value for comparison

\subsection{Equality Operators}
Two of the relational operators are also equality operators (\verb#==#, 
\verb#!=#). Each individual comparison can involve two numbers (including 
\verb#char#), two \verb#boolean# values, or two reference variables. Note that 
you cannot compare incompatible types for equality. When comparing variables 
for equality, \verb#==# and \verb#!=# compare the bit patterns

\subsection{Equality for Primitives}
If a floating-point number is compared with an integer and the values are the 
same, \verb#==# returns \verb#true# as expected

\subsection{Equality for Reference Variables}
Reference variables can be tested to see if they refer to the same object by 
using the \verb#==# operator

\subsection{Equality Gotchas}
\begin{itemize}
    \item Don't mistake \verb#=# for \verb#==# in a boolean expression
    \item Be wary of using \verb#=# rather than \verb#==# within a conditional 
    expression. Remember that the result of any assignment is the value of the 
    variable following the assignment, so the substitution of \verb#=# for 
    \verb#==# in a conditional expression only works for \verb#boolean#s
\end{itemize}

\subsection{Equality for Enums}
Once an enum has been declared, it's not expandable. At runtime there is no way 
to make additional enum constants. You can use either \verb#==# or the 
\verb#equals()# method to test equality of enums

\subsection{instanceof Comparison}
The \verb#instanceof# operator is used for object reference variables only and 
it can be used to check whether an object is of a particular type (i.e. IS-A 
test). If the object being tested is not an actual instantiation of the class 
type on the right side of the operator, \verb#instanceof# will still return 
\verb#true# if the object being compared is assignment compatible with the type 
on the right. Note that \verb#instanceof# can only be used to test an object 
reference against a type from the same class hierarchy -- if a type from a 
different hierarchy is used, compilation will fail. Also note that it is legal 
to test whether \verb#null# is an instance of a class (the result of which is 
always \verb#false#)

\subsection{Arithmetic Operators}
The basic arithmetic operators (\verb#+#, \verb#-#, \verb#*#, \verb#/#) are 
used in the standard way

\subsection{Modulus (\%) Operator}
Divides the left operand by the right and returns the remainder. Remember that 
expressions are evaluated from left to right by default. Parentheses can be 
used to alter the precedence. Also note that \verb#*#, \verb#/# and \verb#%# 
have a higher precedence than \verb#+# and \verb#-#

\subsection{String Contatenation Operator}
The plus sign (\verb#+#) can be used to concatenate two \verb#String#s 
together.  If either operand is a \verb#String#, the \verb#+# operator acts as 
a concatenation operator. If both operands are numbers, it acts as the addition 
operator

\subsection{Increment and Decrement Operators}
Java has two operators that will increment or decrement a variable by exactly 
one. These are composed of either two plus signs (\verb#++#) or two minus signs 
(\verb#--#):
\begin{itemize}
    \item \verb#++# increment (prefix and postfix)
    \item \verb#--# decrement (prefix and postfix)
\end{itemize}
Be wary of using the increment or decrement operators on a \verb#final# 
variable. Any attempt to do so will result in a compilation error

\subsection{Conditional Operator}
A ternary operator used to evaluate a \verb#boolean# expression and then assign 
a value to a variable based on the outcome of the conditional. Its format is: x 
= (\verb#boolean# expression) ? value to assign if \verb#true# : value to 
assign if \verb#false#. Note that the parentheses are optional and conditional 
operators can be nested into one statement

\subsection{Logical Operators}
Note that as of java 1.5, bitwise operators are not on the exam

\subsection{Short-Circuit Logical Operators}
There are five logical operators on the exam that are used to evaluate 
statements that contain more than one \verb#boolean# expression. The most 
commonly used of the five are the two short-circuit operators:
\begin{itemize}
    \item \verb#&&# short-circuit AND
    \item \verb#||# short-circuit OR
\end{itemize}
For an \verb#&&# expression to be \verb#true#, both operands must be 
\verb#true#. A short-circuit \verb#&&# evaluates the left side of the operation 
first and if it resolves to \verb#false#, doesn't bother evaluating the right 
side. For an \verb#||# expression to be \verb#true#, either operand must be 
\verb#true#. A short-cirsuit \verb#||# evaulates the left side of the operation 
first and if it evaluates to \verb#true#, doesn't bother evaluating the right 
side

\subsection{Non-Short-Circuit Logical Operators}
There are two non-short-circuit logical operators:
\begin{itemize}
    \item \verb#&# non-short-circuit AND
    \item \verb#|# non-short-circuit OR
\end{itemize}
These are used in logical expressoins just like \verb#&&# and \verb#||#, but 
don't have the short-circuit functionality -- they always evaluate both sides 
of the expression

\subsection{Logical Operators XOR and NOT}
The last two logical operators on the exam are:
\begin{itemize}
    \item \verb#^# exclusive-OR (XOR)
    \item \verb#!# boolean invert (NOT)
\end{itemize}
The \verb#^# (exclusive-OR) operator evaluates only boolean values and always 
evaluates both operands. For an exclusive-OR expression to be \verb#true#, 
excatly one operand must be \verb#true#. The \verb#!# (boolean invert) operator 
returns the opposite of the boolean argument
