\section{Chapter 3 -- Assignments}
\subsection{Stack and Heap -- Quick Review}
For the most part, the various pieces (methods, variables and objects) of java 
programs live in one of two places -- the stack or the heap:
\begin{itemize}
    \item Instance variables and objects live on the heap
    \item Local variables live on the stack
\end{itemize}

\subsection{Literal Values for All Primitive Types}
\subsubsection{Integer Literals}
There are three ways to represent integer numbers in java -- decimal (base 10), 
octal (base 8) and hexadecimal (base 16)

\subsubsection{Decimal Literals}
Represented as is, with no prefix of any kind

\subsubsection{Octal Literals}
Use only the digits 0 to 7. In java, represent an integer in octal form by 
placing a zero in front of the number. You can have up to 21 digits in an octal 
number, not including the leading zero

\subsubsection{Hexadecimal Literals}
Use digits 0 to 9 and letters a to f. Java will accept lower or upper case for 
the letters. You can have up to 16 digits in a hexadecimal number, not 
including the case insensitive 0x prefix

\subsubsection{Long literals}
Any decimal, octal or hexedecimal integer literals are defined as \verb#int# by 
default, but they can be made \verb#long# by using the case insensitive
extension L

\subsection{Floating-Point Literals}
Defined as a number, a decimal symbol, and more numbers representing the 
fraction. Floating point literals are defined as double (64 bits) by default, 
so if you want to assign a floating-point literal to a variable of type float 
(32 bits), you must attach the suffix F (or f) to the number. If you don't, the 
compiler will complain about a possible loss of precision. You can also 
optionally arrach a D (or d) to double literals, but this is not necessary 
since this is the default debaviour

\subsection{Boolean Literals}
Can either be true or false. Note that numbers can not be used in place of 
boolean values

\subsection{Character Literals}
Represented by a single character in single quotes. Can also use the unicode 
value of the character (prefixing \\u) within single quotes. Remember that 
characters are just 16 bit unsigned integers under the hood

\subsection{Literal Values for Strings}
Although strings are not primitives, they can be represented as literals -- in 
other words, typed directly into code. The only other nonprimitive type that 
has a literal representation is an array.

\subsection{Assignment Operators}
Variables are just bit holders with a designated type. For primitives, the bits 
represent a numeric value (although we don't know what the bit pattern looks 
like for boolean). A reference variable bit holder contains bits representing a 
way to get to an object, or null.

\subsection{Primitive Assignments}
The equals sign is used for assigning a value to a variable. A literal integer 
(such as 7) is always implicitly an \verb#int#. However, the compiler will 
automatically insert the required cast when assigning an integer literal to a 
primitive type of size less than 32 bits. For example, \verb#byte b = 27;# is 
fine and is identical to \verb#byte b = (byte)27;#. The result of any 
expression involving anything \verb#int#-sized or smaller is also an \verb#int#

\subsection{Primitive Casting}
Casting lets you convert primitive values from one type to another. An implicit 
cast means you don't have to write any code for the cast; the conversion 
happens automatically. Typically, an implicit cast happens during a widening 
conversion. The narrowing conversion requires an explicit cast. When casting 
floating-point numbers to integer types, the value loses all the digits after 
the decimal point. Note that when performing an explicit narrowing cast when 
the value being narrowed is too large for the type, the leftmost (i.e.  
higher-order) bits are truncated

\subsection{Assigning Floating-Point Numbers}
Floating-point numbers have slightly different assignment behaviour than 
integer types. Every floating-point literal is implicitly a double (64 bits), 
not a float.

\subsection{Assigning a Literal That is Too Large for the Variable}
If we try to assign a literal which is too large for the variable type, we'll 
get a compilation error, eg. \verb#byte b = 128;#

\subsection{Compound Assignment Operators}
The compound assignment operators (\verb#+=#,\verb#-=#,\verb#*=#,\verb#/=#) all 
insert an implicit cast

\subsection{Assigning One Primitive Variable to Another Primitive Variable}
When you assign one primitive variable to another, the contents of the 
right-hand variable are copied. If we change the contents of either, the other 
variable is not affected

\subsection{Reference Variable Assignments}
You can assign a newly created object to an object referent variable as 
follows: \verb#Button b = new Button();#. This does three things:
\begin{itemize}
    \item Creates a reference variable named \verb#b#, of type \verb#Button#
    \item Creates a new \verb#Button# object on the heap
    \item Assigns the newly created \verb#Button# object to the reference 
    variable \verb#b#
\end{itemize}
You can also assign \verb#null# to a reference variable, which simply means the 
variable is not referring to any object. \verb#Button c = null;#. This creates 
space for the Button reference variable, but doesn't create an actual 
\verb#Button# object. You can assign a subclass of the declared type, but not a 
superclass of the declared type.

\subsection{Variable Scope}
For the purposes of discussing the scope of variables, we can say that there 
are four basic scopes:
\begin{itemize}
    \item Static variables have the longest scope; they are created when the 
    class is loaded, and they survive as long as the class stays loaded in the 
    JVM
    \item Instance variables are the next most long-lived; they are created 
    when a new instance is created and they survive until the instance is 
    removed
    \item Local variables are next, they survive as long as their method 
    remains on the stack
    \item Block variable live only as long as the code block is executing
\end{itemize}
One common scoping error occurs when trying to access a variable which is not 
in scope:
\begin{itemize}
    \item Attempting to access an instance variable from a static context
    \item Attempting to access a local variable from a nested method
    \item Attempting to use a block variable after the code block has completed
\end{itemize}

\subsection{Using a Variable or Array Element That is Uninitialised and 
Unassigned}
\subsubsection{Primitive and Object Type Instance Variables}
Instance variables are initialised to a default value each time a new instance 
is created, although they may be given an explicit value after the 
super-constructors have been completed.

\begin{center}
\begin{tabular}{ll}
    \textbf{Variable Type} & \textbf{Default Value} \\
    \hline
    Object reference & \verb#null# (not referencing any object) \\
    byte, short, int, long & 0 \\
    float, double & 0.0 \\
    boolean & false \\
    char & \verb#'\u0000'# \\
\end{tabular}
\end{center}
Remember that for String reference variables, null is not the same as the empty 
string (\verb#""#)

\subsubsection{Array Instance Variables}
An array is an object, so any array instance variable that is declared but not 
explicitly initialised will have a value of \verb#null#. If the array is 
initialised, all array elements are given their default values -- the same 
default values that elements of that type receive when they are instance 
variables. Array elements are always given default values, regardless of where 
the array itself is declared or instantiated

\subsection{Local (Stack, Automatic) Primitives and Objects}
Local variables, including primitives must always be initialised before they 
are used (though not necessarily in the same line of code). Java does not give 
local variables a default value. The compiler can't always tell whether a local 
variable has been initialised before use. For example, if you initialise within 
a conditional block (in other words, a code block that may not be run, such as 
an \verb#if# block or \verb#for# loop without a literal value of \verb#true# or 
\verb#false# in the test), the compiler knows that the initialisation might not 
happen and can produce an error

\subsection{Local Object References}
Local reference variables are not given a default value -- that is, they aren't 
\verb#null#. If you don't initialise a local reference variable, then by 
default is doesn't have any value

\subsection{Local Arrays}
Just like any other object reference, array references declared with a method 
must be assigned a value before use. This means you must declare and construct 
the array, but you don't have to explicitly initialise the elements in the
array -- remember that the elements of the array are automatically initialised 
anyway

\subsection{Assigning One Reference Variable to Another}
With primitives, an assignment of one variable to another means the contents 
(bit pattern) of one variable are copied into another. Object refernce variable 
work exactly the same way -- the contens of the reference variable (a bit 
pattern) are copied to the other reference variable. One exception to the way 
object references are assigned is String. String objects are given special 
treatment -- they are immutable. For another other type, where two references 
refer to the same object, if either reference is used to modify the object, 
both references will see the change because there is still only one object.  
However, any time we change a String, the JVM will update the reference 
variable to refer to a different object. The different object might be a new 
object, or it might not, but it will definitely be a different object.

When you use a String reference variable to modify a String:
\begin{itemize}
    \item A new String is created (or a matching String is found in the String 
    pool), leaving the original String object untouched
    \item The reference used to modify the String (or rather, make a new String 
    String by modifying a copy of the original) is then assigned the new String 
    object
\end{itemize}

\subsection{Passing Variables into Methods}
\subsubsection{Passing Object Reference Variables}
When you pass an object reference variable into a method, you must keep in mind 
that you're passing the object reference, and not the object itself. More 
importantly, remember that you aren't even passing the actual reference 
variable, but a copy of the reference variable

\subsection{Java and Pass-By-Value Semantics}
It makes no difference if passing a primitive or reference variable, you are 
always passing a copy of the bits in the variable. The bottom line on 
pass-by-value: the called method can't change the caller's variable, although 
for object reference variables, the called method can change the object the 
    variable referred to. For object references, this means the called method 
    can't reassign the caller's original reference variable and make it refer 
    to a different object, or \verb#null#

\subsection{Array Declaration, Construction and Initialisation}
\subsubsection{Declaring an Array}
Arrays are declared by stating the type of element teh array will hold, which 
can be an object or a primitive, followed by square brackets to the left or 
right of the identifier. It is never legal to include the size of the array in 
the declaration. The JVM doesn't allocate space until you actually instantiate 
the array object

\subsubsection{Constructing an Array}
Constructing an array means creating the array object on the heap, where all 
objects (and instance variables) live. To create an array object, java must 
know how much space to allocate on the heap, so you must specify the size of 
the array at construction time.

\subsubsection{Constructing 1-D Arrays}
The most straightforward way to  construct an array is to use the keyword 
\verb#new# followed by the array type, with a bracket specifying how many 
elements of that type the array will hold. Remember that when constructing an 
array of object references, the object's constructor is not invoked. Note that 
the words `construct', `create' and `instantiate' all refer to an object being 
created on the heap. Note that due to syntax shortcuts, objects can be created 
without even invoking \verb#new#

\subsubsection{Constructing Multidimensional Arrays}
Multidimensional arrays are simply arrays of arrays. So a 2D array of type int 
is really an object of type int array (\verb#int[]#), with each element in the 
array holding a reference to another int array. Note that only the first 
dimension is required when constructing the array and the other dimensions need 
not all be the same size

\subsubsection{Initialising an Array}
Initialising an array means putting things into it. The `things' in the array 
are the array's elements, and they're either primitives or reference variables.
If you have an array of objects, the array doesn't actually hold the object, 
but instead holds a reference to the object. The individual elements in the 
array can be accessed with an index number. The index number always begin with 
zero. Note that an \verb#ArrayIndexOutOfBoundsException# will be thrown at 
runtime for invalid index values -- compilation will not fail even if using an 
invalid index

\subsubsection{Initialising Elements in a Loop}
Array objects have a single public variable, \verb#length#, that gives the 
number of elements in the array

\subsubsection{Declaring, Constructing and Initialising on One Line}
There are two array-specific syntax shortcuts to both construct (instantiate 
the array object itself) and initialise (put explicit values into an array's 
elements) in a single statement. The first is used to declare, create and 
intialise as follows: \verb#int x = 9; int[] a = {6,x,8};# You can also use 
this syntax shortcut with multidimensional arrays

\subsubsection{Constructing and Initialising an Anonymous Array}
The second shortcut is called `anonymous array creation' and can be used to 
construct and initialise an array, and then assign the array to a previously 
declared array reference variable: \verb#int[] a; a = new int[] {4,7,2};#.  
Remember that it is illegal to speicify a size when creating an anonymous array

\subsubsection{Legal Array Element Assignments}
\subsubsection{Arrays of Primitives}
Primitive arrays can accept any value that can be implicitly promoted to the 
declared type of the array

\subsubsection{Arrays of Object References}
If the declared array type is a class (or interface), you can put objects of 
any subclass of the declared type (or implementing class) into the array

\subsubsection{Array Reference Assignments for 1-D Arrays}
If you declare a primitive array, the reference variable you declared can be 
reassigned to any array of the same type (of any size), but cannot be 
reassigned to any other type of array (or anything else). For object reference 
arrays, the array reference can be assigned to an array of any subclass of the 
declared class (or an implementing class in cases of interface type arrays)

\subsubsection{Array Reference Assignments for Multidimensional Arrays}
When assigning an array to a previously declared array reference, the array 
you're assigning must be the same dimension as the refernce being assigned to.  
Note to be careful with array assignments using different dimensions

\subsection{Initialisation Blocks}
Initialisation blocks are the third place in a java program where operations 
can be performed (the others being constructors and methods). They run when the 
class is first loaded (if static) or when an instance is created (if instance).  
The syntax is fairly simple -- they don't have names, can't take arguments and 
don't return antyhing. The rules regarding initialisation blocks are:
\begin{itemize}
    \item Init blocks execute in the order they appear
    \item Static init blocks run once, when the class is first loaded
    \item Instance init blocks run every time a class instance is created
    \item Instance init blocks run after the constructor's call to 
    \verb#super()#
\end{itemize}
Instance init blocks are often used as a place to put code that all the 
constructors in the class should share. Note that an 
ExceptionInitializationError can be thrown from within a static initialisation 
block

\subsection{Using Wrapper Classes and Boxing}
The wrapper classes in the java api serve two main purposes:
\begin{itemize}
    \item To a provide a means to `wrap up' primitive values in an object so 
    they can be used in activities reserved for objects
    \item To provide an assortment of utility functions for primitives -- 
    mostly related to various conversions
\end{itemize}

\subsection{An Overview of the Wrapper Classes}
There are wrapper classes for every primitive in java. The wrapper name is just 
the capitalised form of the primitive except for \verb#char# (which is 
\verb#Character#) and \verb#int# (which is \verb#Integer#)

\subsection{Creating Wrapper Objects}
Creation methods which take a String throw a NumberFormatException if the 
String provided cannot be parsed into the appropriate primitive. Note that 
wrapper objects are immutable

\subsubsection{Wrapper Constructors}
All of the wrapper classes except Character provide two constructors -- one 
that takes a primitive, and one that takes a String. The Character class 
provides only one constructor which takes a char. The constructors for 
Boolean's String constructor return the object `true' if the String argument is 
case insensitive the same as `true', otherwise `false' is returned. Note that 
as of java 5m a Boolean object can be used in a boolean test (due to unboxing)

\subsubsection{The valueOf() Methods}
The static \verb#valueOf()# methods provided in most of the wrapper classes 
provide another means of creating wrapper objects. They take a String 
representation of the primitive and sometimes a second argument specifying the 
radix of the first argument. Note that a NumberFormatException may be thrown

\subsection{Using Wrapper Conversion Utilities}
\subsubsection{xxxValue()}
When converting the value of a wrapped numeric to a primitive, use one of the
\verb#xxxValue()# methods. 

\subsubsection{parseXxx() and valueOf()}
The six \verb#parseXxx()# methods (one for each numeric wrapper type) are 
closely related to the \verb#valueOf()# method. Both take a String argument, 
throw a NumberFormatException and can convert String objects from different 
bases. The difference between them is:
\begin{itemize}
    \item \verb#parseXxx()# returns the named primitive
    \item \verb#valueOf()# returns a newly created wrapper object of the type 
    that invoked the method
\end{itemize}

\subsection{toString()}
All of the wrapper classes classes have a no-arg, non-static instance version 
of \verb#toString()#. This method returns a String with teh value of the 
primitive wrapped in the object. All of the numeric wrapper classes provide an 
overloaded, static \verb#toString()# method that takes a primitive numeric of 
the appropriate type and returns a String. Integer and Long provide a third 
\verb#toString()# method which is static and accepts the primitive and a radix

\subsection{toXxxstring() (Binary, Hexadecimal and Octal)}
The Integer and Long wrapper classes provide static \verb#toXxxString()# 
methods which take a primitive and return the String representation in the 
appropriate radix.

\subsection{Wrapper Conversion Summary}
In summary, the essential method signatures for wrapper conversion methods are:
\begin{description}
    \item[primitive xxxValue()] Converts a wrapper to a primitive
    \item[primitive parseXxx(String)] Converts a String to a primitive
    \item[Wrapper valueOf(String)] Converts a String to a wrapper
\end{description}

\subsection{Autoboxing}
As of java 5, java can automatically box and unbox wrapper objects. For 
example, this means you can do this: \verb#Integer y = new Integer(5); y++;# 
Note that this is simply a convenience -- behind the scenes, the compiler 
inserts the boxing and unboxing for you and a second object is created (since 
wrapper objects are immutable)

\subsection{Boxing, == and equals()}
In order to save memory, two instance of the following wrapper obejcts (created 
through boxing), will always be \verb#==# when their primitive values are the 
same:
\begin{itemize}
    \item Boolean
    \item Byte
    \item Character from \\u0000 to \\u007f (7f is 127 in decimal)
    \item Short and Integer from -128 to 127
\end{itemize}
Note that when \verb#==# is used to compare a primitive to a wrapper, the 
wrapper will be unwrapped and the comparison will be primitive to primitive

\subsection{Where Boxing Can be Used}
The general rule is that boxing and unboxing will work whereever you can 
normally use a primitive or a wrapped object. Note that wrapper reference 
variables can be \verb#null# -- be aware of primitive operations which may 
throw a NullPointerException if the wrapper object is null

\subsection{Overloading}
\subsubsection{Primitive Arguments}
When an exact match isn't found, teh JVM uses the method with the smallest 
argument that is wider than the parameter

\subsubsection{Boxing and Var-Args}
The compiler will choose widening over boxing (this is to maintain backwards 
compatibility). In summary:
\begin{itemize}
    \item Widening beats boxing
    \item Boxing beats var-args
\end{itemize}

\subsubsection{Widening Reference Variables}
Reference widening depends on inheritance (i.e. the IS-A test). Therefore, it 
is not legal to widen from one wrapper to another

\subsubsection{Overloading When Combining WIdening and Boxing}
It is not possible to widen and then autobox. However, it is possible to box 
and then widen (through inheritance)

\subsubsection{Overloading in Combination with Var-Args}
Var-args can be combined with either widening or boxing

\subsubsection{Widening, Boxing and Var-Args Summary}
\begin{itemize}
    \item Primitive widening uses the `smallest' method argument possible
    \item Used individually, boxing and var-args are compatible with 
    overloading
    \item It is not possible to widen from one wrapper to another (IS-A fails)
    \item It is not possible to widen and then box (an \verb#int# can't become 
    a \verb#Long#)
    \item Is it possible to box and then widen (an \verb#int# can become an 
    \verb#Object#)
    \item Var-args can be combined with either widening or boxing
\end{itemize}

\subsection{Garbage Collection}
Java's garbage collector provides an automatic solution to memory management.  
In most cases, it frees you from having to add memory management logic to your 
application. The downside of automatic garbage collection is that you cannot 
control when it runs. Objects live in the heap and this is the only part of 
memory that is involved with garbage collection. When the garbage collector 
runs, its purpose is to find and delete objects that cannot be reached.

\subsubsection{When does the Garbage Collector Run?}
The garabge collector is under the control of the JVM and it decides when to 
run it. We can ask the JVM to run the garbage collector, but there are no 
guarantees that it will run when 

\subsubsection{How does the Garbage Collector Work}
You can't be sure -- the java specification doesn't guarantee any particular 
implementation. An object is eligible for garbage collection when no live 
thread can access it. If object A has a reference to object B and object A is 
available to a live thread, then object B is considered reachable. Note that 
due to the vagaries of the String constant poll, the exam focuses its garbage 
collection questions on non-String objects

\subsubsection{Writing Code that Makes Objects Eligible for Garbage Collection}
There are a few ways to remove a reference to a reference variable:
\begin{itemize}
    \item Nulling a reference is
    \item Reassigning a reference variable
    \item Objects created within a method (whose reference is not returned from 
    the method) are eligible for collection once the method has returned
    \item Objects may form `islands of isolation' whereby objects refer to one 
    one another but cannot be reached by any live thread. The garbage collector 
    can usually discover and remove these islands
\end{itemize}

\subsubsection{Requesting Garbage Collection}
The garbage collection routines that java provides are members of hte Runtime 
class. The Runtime class is a special class that has a singleton for each main 
program. You can get the singleton through \verb#Runtime.getRuntime()# and then 
request garbage collection through \verb#gc()#. Alternatively, you can call 
\verb#System.gc()# which does the same thing. 

\subsubsection{Cleaning Up Before Garbage Collection -- the finalize() Method}
Java provides a mechanism to run some code just before an object is deleted by 
the garbage collector. This code is located in the \verb#finalize()# method 
that all classes inherit from \verb#Object#. At first, this seems like a good 
idea, however since the garbage collector might not ever run, there is no 
guarantee that the \verb#finalize()# method runs. Also, not that:
\begin{itemize}
    \item For any given object, \verb#finalize()# will be called at most once 
    by the garbage collector
    \item The finalize method can result in an object being saved from deletion
\end{itemize}
