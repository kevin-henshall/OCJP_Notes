\section{Chapter 1 -- Declarations and Access Control}
\subsection{Legal Identifiers}
\begin{itemize}
    \item Identifiers must start with a letter, currency character or a 
    connecting character (eg. underscore but not minus)
    \item Following the first character, identifiers may contain any 
    combination of currency characters, connecting characters or digits
\end{itemize}

\subsection{JavaBeans Standards}
Property Naming Rules
\begin{itemize}
    \item If the property is not a boolean, the getter's method prefix must be 
    \verb#get#
    \item If the property is a boolean, the getter's method prefix must be 
    either \verb#get# or \verb#is#
    \item The setter's method prefix must be \verb#set#
    \item The setter's method must be \verb#public void# and take an argument 
    type which represents the property type
    \item The getter's method must be \verb#public#, take no arguments and have 
    a return type which matches the argument type of the setter
\end{itemize}
Listener Naming Rules
\begin{itemize}
    \item Listener methods used to ``register" a listener must use the prefix 
    \verb#add#, followed by the listener type
    \item Listener methods used to ``unregister" a listener must use the prefix 
    \verb#remove#, followed by th listener type
    \item The type of listener must be passed as the argument to the method
\end{itemize}

\subsection{Source File Declaratoin Rules}
\begin{itemize}
    \item There can be at most one \verb#public# class per file
    \item If there is a public class, the filename must match the class name
    \item Package statement must be before any imports
    \item Import statements must be before the class declaration
    \item Import and package statements apply to all classes within the file
    \item A file may have more than one non-public class
    \item A file with no public class may have a name that does not match any 
    of the classes declared within it
\end{itemize}

\subsection{Class Declarations and Modifiers}
There are four access controls (levels of access):
\begin{description}
    \item[public] All classes from all packages have access
    \item[protected]
    \item[private]
    \item[default] No modifier preceeding the class declaration. Can be thought 
    of as package-level access
\end{description}

\subsection{Other (Non-Access) Class Modifiers}
Class declaractions can be modified using the kaywords \verb#final#, 
\verb#abstract# or \verb#strictfp# . These modifiers are in addition to the 
access modifiers already present, however not all combinations of non-access 
and access modifiers can be used (eg. \verb#final# and \verb#abstract#).
\begin{description}
    \item[strictfp] Is a modifier which can be applied to a class or method 
    (not variable) and indicates that the code conforms to the IEEE 754 
    standard for floating point numbers. Without this modifier, the code may 
    behave in a platform dependent manner
    \item[final] A \verb#final# class can not be subclassed
    \item[Abstract] An \verb#abstract# class can not be instantiated -- it's 
    sole purpose is to be extended (subclassed). Note that methods can also be 
    marked as \verb#abstract#, in which case, they are declared (not defined) 
    in the abstract class (be wary of method definitions within abstract 
    classes). Also note that if a method is marked as abstract, then the class 
    must be marked as abstract too, however not all methods need be 
    \verb#abstract# within an \verb#abstract# class
\end{description}

\subsection{Declaring Interfaces}
An interface can be thought of as a 100\% abstract class, however an interface 
can have only abstract methods. There are strict rules regarding interfaces:
\begin{itemize}
    \item All interface methods are implicitly \verb#public# and 
    \verb#abstract# (note that the modifiers can appear in any order)
    \item All variables are implicitly \verb#public#, \verb#static# and 
    \verb#final# (note that the modifiers can appear in any order)
    \item Methods cannot be marked \verb#static#
    \item Since methods are \verb#abstract#, they cannot be marked 
    \verb#final#, \verb#strictfp# or \verb#native#
    \item An interface can extend multiple interfaces
    \item An interface can only extend other interfaces
    \item An interface cannot cannot implement another interface or class
    \item An interface must be declared with the keyword \verb#interface#
    \item Interface types can be use polymorphically
\end{itemize}

\subsection{Class Member Access Modifiers}
Class members can use all four access modifiers:
\begin{description}
    \item[public] All other classes, regardless of the package they belong to, 
    can access the member (assuming the class is visible). For a subclass, if a 
    member of its superclass is declared \verb#public#, the subclass inherits 
    that member regardles of package
    \item[protected] Can be accessed through inheritence by classes in a 
    different package. Think package + kids (where kids are subclasses). Note 
    that once a subclass-outside-the-package inherits a \verb#protected# 
    member, that member (as inherited by the subclass) becomes private to any 
    code outside the subclass, with the exception of subclasses of the subclass
    \item[default] Can only be accesed through inheritance by classes within 
    the same package. Think package restriction
    \item[private] Can't be accessed by any class other than the class in which 
    the \verb#private# member was declared. Note that private members are not 
    inherited and cannot be overridden
\end{description}

\subsection{Local Variables and Access Modifiers}
There is never a case where an access modifier can be applied to a local 
variable. There is only one modifier which can be applied to local variables -- 
\verb#final#

\subsection{Determining Access to Class Members}
\begin{center}
\begin{tabular}{lcccc}
    \textbf{Visibility} & \textbf{Public} & \textbf{Protected} & 
    \textbf{Default} & \textbf{Private} \\
    \hline
    From the same class & Yes & Yes & Yes & Yes \\
    \hline
    From any class in the same package & Yes & Yes & Yes & No \\
    \hline
    From a subclass in the same package & Yes & Yes & Yes & No \\
    \hline
    From a subclass outside the same package & Yes & Yes (through inheritence) 
    & No & No \\
    \hline
    From any non-subclass class outside the package & Yes & No & No & No \\
    \hline
\end{tabular}
\end{center}

\subsection{Final Methods}
The \verb#final# keyword prevents a method from being subclassed. Preventing a 
subclass from overriding a method provides safety and security but also stifles 
many of the benefits of OO including extensibility and polymorphism

\subsection{Final Arguments}
Method arguments are essentially the same as local variables -- final arguments 
follow the same rules as final local variables.

\subsection{Abstract Methods}
An abstract method is a method which is declared (as abstract) but not 
implemented. Note that an abstract method declaration must end in a semicolon 
and contains no code -- not even \verb#{}#. Note that abstract methods can only 
be declared in abstract classes or interfaces. However, an abstract class need 
not have any abstract methods. Any class that extends an abstract class must 
implement all abstract methods of the superclass, unless teh subclass is also 
abstract. The rule is:
\begin{itemize}
    \item The first concrete subclass of an abstract class must implement all 
    abstract methods of the superclass
\end{itemize}
Note that if the subclass is abstract, there is no need to re-declare an 
abstract method of the superclass as being abstract. Additional notes regarding 
abstract classes and modifiers:
\begin{itemize}
    \item A method can never be marked as both \verb#abstract# and 
    \verb#final#, or both \verb#abstract# and \verb#private#
    \item The \verb#abstract# modifier can never be combined with the 
    \verb#static# modifier
\end{itemize}

\subsection{Synchronized Methods}
The \verb#synchronized# modifier indicates that a method (not variable) can be 
accessed by only one thread at a time

\subsection{Native Methods}
The \verb#native# modifier indicates that a method is implemented in 
platform-dependent code, often in C. Note that \verb#native# methods are only 
declared (not defined)

\subsection{Strictfp Methods}
The \verb#strictfp# modifier can be applied to both classes and methods to 
indicate that the class or method conforms to the IEEE 754 standard. Note that 
a method can be declared as \verb#strictfp# even if the enclosing class is not

\subsection{Var-args}
Declaration rules for var-args:
\begin{description}
    \item[Var-arg type] Must specify the type of argument(s) this parameter can
    receive
    \item[Syntax] Follow the type with an ellipses (\ldots) and then the name 
    of the array which will hold the parameters received
    \item[Other parameters] Other parameters may be specified
    \item[Var-arg limits] The var-arg must be the last parameter in the
    method's signature and only one var-arg parameter can be 
    specified
\end{description}

\subsection{Constructors}
Constructors must have the same name as the class in which they are defined.
Constructors never have a return type, however than can have all of the normal 
access modifiers for methods, and can take arguments (including var-args. They 
cannot be marked \verb#static#, \verb#final# or \verb#abstract#

\subsection{Variable Declarations}
There are two types of variables in java:
\begin{description}
    \item[Primitives] Can be one of the eight types: \verb#char#, 
    \verb#boolean#, \verb#byte#, \verb#short#, \verb#int#, \verb#long#, 
    \verb#float# or \verb#double#. Once a primitive is declared, its type can 
    never change
    \item[References] Used to refer to (or access) an object. They are declared 
    with a specific type and can never change. A reference can be used to refer 
    to any object of the declared type, or a subtype of the declared type
\end{description}

\subsection{Primitive sizes}
\verb#boolean# primitives do not have a range - only true or false. The bit 
depth of a boolean is virtual machine dependent. \verb#char# contains a 16 bit 
(unsigned) unicode character, meaning that a char can be assigned to anything 
larger than a \verb#short#

\begin{center}
\begin{tabular}{lll}
    \textbf{Type} & \textbf{Bits} & \textbf{Bytes} \\
    \hline
    \verb#byte# & 8 & 1 \\
    \verb#short# & 16 & 2 \\
    \verb#int# & 32 & 4 \\
    \verb#long# & 64 & 8 \\
    \verb#float# & 32 & 4 \\
    \verb#double# & 64 & 8 \\
\end{tabular}
\end{center}

\subsection{Instance Variables}
Instance variables are defined within the class, but outside of any method and 
are only initialised when the class is instantiated. Instance variables are the 
fields which belong to each unique object. Instance variables:
\begin{itemize}
    \item Can use any of the four access levels
    \item Can be marked as \verb#final#
    \item Can be marked \verb#transient#
    \item Cannot be marked \verb#abstract#
    \item Cannot be marked \verb#synchronized#
    \item Cannot be marked \verb#strictfp#
    \item Cannot be marked \verb#native#
    \item Cannot be marked \verb#static# -- in this case, they would be class 
    variables
\end{itemize}

\subsection{Summary of Modifiers}
\begin{center}
\begin{tabular}{ccc}
    \textbf{Local Variables} & \textbf{Variables (non-local)} & 
    \textbf{Methods} \\
    \hline
    final & final & final \\
    & public & public \\
    & protected & protected \\
    & private & private \\
    & static & static \\
    & transient & \\
    & volitile & \\
    & & abstract \\
    & & synchronized \\
    & & strictfp \\
    & & native \\
\end{tabular}
\end{center}

\subsection{Local (Automatic/Stack/Method) Variables}
Local variables are variables within a method. They are in scope only for the 
duration of the block in which they are defined and always reside on the stack, 
not the heap -- even for local reference variables. Remember that there is no 
such thing as a stack object, only a stack variable. Although the value of a 
variable may be passed to another method that then stores the value, the 
variable itself is only alive for the scope in which it was defined. Local 
variables cannot utilise most of the modifiers with the exception of 
\verb#final#. Note that a local variable must be initialised before it can be 
used, otherwise compilation will fail, since local variables do not get default 
values (unlike instance variables). It is possible to declare a local variable 
with the same name as an instance variable (known as shadowing)

\subsection{Array Declarations}
Arrays are objects that store multiple variables of the same type, or variables 
that are all subclasses of the same type. Arrays can hold either primitives or 
or object references, but the array will always be an object on teh heap, even 
if it contains primitives

\subsection{Array Declaration}
Arrays are declared by stating the type of elements the array will hold, 
followed by square brackets to either side of the identifier. Arrays can be 
multidimensional (arrays of arrays), in which case, the square brackets can be 
mixed between the left and right of the identifier. Note that it is never legal 
to include the size of an array in the declaration.

\subsection{Final Variables}
Decalring a variable as \verb#final# makes it impossible to reinitialise once 
it has been explicitly initialised. A reference variable marked as \verb#final# 
cannot be reassigned to refer to another object. The data within the object can 
be modified, but the reference cannot be changed. Remember that there are no 
final objects, only final references. Note that final instance variables must 
be assigned before the constructor completes

\subsection{Transient Variables}
Instance variables can be marked as \verb#transient#, meaning that the JVM will 
ignore this variable when attempting to serialise the object containing it.

\subsection{Volatile Variables}
Instance variables can be marked as \verb#volatile#, meaning that a thread 
accessing the variable must always recolcile its own private copy of the 
variable with the master copy in memory

\subsection{Static Variables and Methods}
The \verb#static# modifier is used to create variables and methods that will 
exist independently of any instances of the class. All \verb#static# members 
exist once the class is loaded, and there will only be one copy of each, 
regardless of the number of instances of that class. Things which can be marked 
as \verb#static#:
\begin{itemize}
    \item Methods
    \item Variables
    \item A class nested within another class, but not within a method
    \item Initialization blocks
\end{itemize}
Things which cannot be marked as \verb#static#:
\begin{itemize}
    \item Constructors
    \item Classes (unless they are nested)
    \item Interfaces
    \item Method local inner classes
    \item Inner class methods and instance variables
    \item Local variables
\end{itemize}

\subsection{Declaring Enums}
Enums can be declared as their own separate class, or as a class member, 
however they must not be declared within a method. An enum that isn't enclosed 
in a class can be declared with only the \verb#public# or default modifier, 
just like a non-inner class. Note that the syntax for accessing an enum's 
members depends on where the enum was declared (i.e. if declared in a class, 
the enclosing class name is required when accessing the enum's members). Note 
that it is optional to place a semicolon following an enum declaration. One can 
iterate through the values of an enum by invoking the values() method on any 
enum type

\subsection{Declaring Constructors, Methods and Variables in an Enum}
Since an enum is really a special kind of class, you can do more than just list 
the enumerated constant values. You can add constructors, instance variables, 
methods and also constant specific class bodies. The key points to remember 
about enum constructors are:
\begin{itemize}
    \item An enum constructor can never be invoked directly -- it is invoked 
    automatically, with the arguments you define after the constant value
    \item You can define multiple arguments, and overload the enum constructor
\end{itemize}
Constant spcific class bodies can be used to override a method for a specific 
constant defined in the enum
