\section{Chapter 8 -- Inner Classes}
Inner classes allow you to define one class within another. They provide a type 
of scoping for your classes since you can make one class a member of another 
class. Just as classes have member variables and methods, a class can also have 
member classes. They come in several varieties, depending on how and where you 
define the inner class

\subsection{Inner Classes}
Sometimes you find yourself designing a class where you discover you need 
behaviour that belongs in a separate, specialised class, but also needs to be 
intimately tied to the class you are designing. Inner classes address this 
problem.

One of the key benefits of an inner class is the `special relationship' an 
inner class instance shares with an instance of the outer class. This 
relationship gives code in the inner class access to members of the enclosing 
(outer) class, as if the inner class were part of the outer class. In fact, that 
is exactly what it means -- the inner class is a part of the outer class. Not 
just a `part', but a member of the outer class. An inner class instance has 
access to all members of the outer class, even those marked private

\subsubsection{Coding a `Regular' Inner Class}
We use the term `regular' to represent inner classes that  are not:
\begin{itemize}
    \item Static
    \item Method-local
    \item Anonymous
\end{itemize}
You define an inner class within the curly braces of the outer class:
\begin{verbatim}
    class Outer {
        class Inner {}
    }
\end{verbatim}
Compiling results in two class files: \verb#Outer.class# and 
\verb#Outer$Inner.class#. In the end, the inner class is still a separate 
class, so a separate class file is generated for it. However, the inner class 
file is not accessible in the usual way. You can not say \verb#java Outer$Inner# in the hope of running the \verb#main()# method of the inner 
class, because a regular inner class can not have static declarations of any 
kind. The only way you can access the inner class is through a live instance of 
the outer class.

Just as any member of the outer class (eg. an instance method) can acess any 
other member of the outer class (\verb#private# or otherwise), the inner class 
can do the same

\subsubsection{Instantiating an Inner Class}
To create an instance of an inner class, you must have an instance of the outer 
class to tie to the inner class. There are no exceptions to this rule: an inner 
class can never stand alone without a direct relationship to an instance of the 
outer class.

Summary of the differences between inner class instantiation code that is 
within the outer class (but not static), and inner class instantiation code 
that is outside the outer class:
\begin{itemize}
    \item From within the outer class instance code, use the inner class name 
    in the normal fashion:
        \begin{verbatim}
            Inner inner = new Inner();
        \end{verbatim}
    \item From outside the outer class instance code (including static method code 
    within the outer class), the inner class name must include the outer 
    class's name: \verb#Outer.Inner#. To instantiate it, you must use a 
    reference to the outer class:
        \begin{verbatim}
            Outer.Inner inner = new Outer().new Inner();
        \end{verbatim}
    or
        \begin{verbatim}
            Outer outer = new Outer();
            Outer.Inner inner = outer.new Inner();
        \end{verbatim}
\end{itemize}

\subsubsection{Referencing the Inner or Outer Instance from Within the Inner 
Class}
A quick review of \verb#this#:
\begin{itemize}
    \item The keyword \verb#this# can be used only from within instance code -- 
    not within \verb#static# code
    \item The \verb#this# reference is a reference to the currently executing 
    object. In other words, the object whose reference was used to invoke the 
    currently running method
    \item The \verb#this# reference is the way an object can pass a reference 
    to itself to some other code as a method argument
\end{itemize}

The rules for an inner class referencing itself or the outer instance are:
\begin{itemize}
    \item To reference the inner class itself from within the inner class code, 
    use \verb#this#
    \item To reference the `outer \verb#this#' (the outer class instance) from 
    within the inner class code, use \verb#OuterClassName.this#
\end{itemize}

\subsubsection{Member Modifiers Applied to Inner Classes}
A regular inner class is a member of the outer class just as instance variables 
and methods are, so the following modifiers can be applied to an inner class:
\begin{itemize}
    \item \verb#final#
    \item \verb#abstract#
    \item \verb#public#
    \item \verb#private#
    \item \verb#protected#
    \item \verb#strictfp#
    \item \verb#static# -- but \verb#static# turns it into a \verb#static# 
    nested class, not an inner class
\end{itemize}

\subsection{Method-Local Inner Classes}
A regular inner class is scoped inside another class's curly braces, but 
outside any method code (in other words, at the same level that an instance 
variable is declared). However, you can also define an inner class within a 
method.

To use the inner class you must make an instance of it somewhere within the 
method but after the inner class definition (or the compiler will not be able 
to find the inner class). The following legal code shows how to instantiate and 
use a method-local inner class:
\begin{verbatim}
    class Outer {
        private String x = "Outer";
        void doStuff() {
            class Inner {
                public void seeOuter() {
                    System.out.println("Outer x is " + x);
                }
            }
            Inner inner = new Inner();  // must instantiate after class definition
            inner.seeOuter();
        }
    }
\end{verbatim}

\subsubsection{What a Method-Local Inner Class Object Can and Cannot Do}
A method-local inner class can be instantiated only within the method where the 
inner class is defined. In other words, no other code running in any other 
method (inside or outside the outer class) can ever instantiate the 
method-local inner class. Like regular inner class objects, the method local 
inner class object shaers a special relationship with the enclosing (outer) 
class object, and can access its \verb#private# (or any other) members.  
However, the inner class object cannot use the local variables of the method 
the inner class is in, unless the local variables are marked \verb#final#.

Note that the only modifiers you can apply you a method-local inner class are 
\verb#abstract# and \verb#final#, though never both at the same time. Note that 
a method-local inner class defined within a \verb#static# method has access to 
only \verb#static# members of the enclosing class, since there is no associated 
instance of the enclosing class

\subsection{Anonymous Inner Classes}
An anonymous inner class is an inner class without any class name at all. They 
can be defined within a method, or within an argument to a method

\subsubsection{Plain-Old Anonymous Inner Classes, Flavour One}
\begin{verbatim}
    class Popcorn {
        public void pop() {
            System.out.println("popcorn");
        }
    }
    class Food {
        Popcorn p = new Popcorn() {
            public void pop() {
                System.out.println("anonymous popcorn");
            }
        };          // Note: semicolon required
    }
\end{verbatim}
In the above, the \verb#Popcorn# reference variable does not refer to an 
instance of \verb#Popcorn#, but to an instance of an anonymous (unnamed) 
subclass of \verb#Popcorn#. The idea behind this is to override one or more 
methods of the superclass (or to implement an interface). Note that you can 
only call methods on an anonymous inner class reference that are defined in the 
reference variable type. If you attempt to invoke a method which is not defined 
in the reference variable type, compilation will fail (even if the method is 
defined in the anonymous inner class)

\subsubsection{Plain-Old Anonymous Inner Classes, Flavour Two}
The only difference between flavour one and flavour two is that flavour one 
creates an anonymous subclass of the specified type, whereas flavour two 
creates an anonymous implementer of the specified interface type
\begin{verbatim}
    interface Cookable {
        public void cook();
    }
    class Food {
        Cookable c = new Cookable() {
            public void cook() {
                System.out.println("anonymous cookable implementer");
            }
        };          // Note: semicolon required
    }
\end{verbatim}
Note that this is the only time you will see the syntax \verb#new Cookable()# 
where \verb#Cookable# is an interface rather than a non-abstract class type.

Note that anonymous interface implementers can implement only one interface.  
There simply is not a mechanism that allows an anonymous inner class to 
implement multiple interfaces. In fact, an anonymous inner class can not even 
extend a class and implement an interface at the same time. The inner class has 
to choose either to be a subclass of a named class -- and not directly (by 
using the keyword \verb#implements#) implement any interfaces at all -- or to 
implement a single interface. If the anonymous inner class is a subclass of a 
class type, it automatically becomes an implementer of any interfaces 
implemented by the superclass

\subsubsection{Argument-Defined Anonymous Inner Classes}
It is possible to define an anonymous inner class as an argument to a method.  
In this case, pay special attention to the syntax used to end the method 
invocation

\subsection{Static Nested Classes}
You will sometimes hear static nested classes referrred to as static inner 
classes, but they are not inner classes at all by the standard definition of an 
inner class. While an inner class (regardless of the flarour) has a special 
relationship with instances of the outer class, a static nested class does not.
It is simply a non-inner (also called `top-level') class scoped within another.
So with static classes, it is really more about namespace resolution than an 
implicit relationship between the two classes.

A static nested class is simply a class that is a static member of the 
enclosing class:
\begin{verbatim}
    class Outer {
        static class Inner {}
    }
\end{verbatim}
The class itself is not really `static' -- there is no such thing as a static 
class. The \verb#static# modifier in this case means that the class is a static 
member of the outer class. This means it can be accessed (as with other static 
members), without having an instance of the outer class

\subsubsection{Instantiating and Using Static Nested Classes}
You use standard syntax to access a static nested class from its enclosing 
class. The syntax for instantiating a static nested class from a non-enclosing 
class is a little different from a normal inner class:
\begin{verbatim}
    class OuterA {
        static class InnerA {
            void go() { System.out.println("hiA"); }
        }
    }
    class OuterB {
        static class InnerB {
            void go() { System.out.println("hiB"); }
        }
        public static void main(String[] args) {
            OuterA.InnerA a = new OuterA.InnerA();      // both class names, one `new'
            a.go();                                     // produces "hiA"
            InnerB b = new InnerB();                    // only one class name
            b.go();                                     // produces "hiB"
        }
    }
\end{verbatim}
Just as a static method does not have access to the instance variables and 
non-static methods of a class, a static nested class does not have access to 
the instance variables and non-static methods of the outer class
