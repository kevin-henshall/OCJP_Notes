\section{Chapter 5 -- Flow Control, Exceptions and Assertions}
\subsection{if and switch Statements}
\subsubsection{if-else Branching}
The expression in an \verb#if# statement must be a \verb#boolean# expression. 
The basic format is:
\begin{verbatim}
    if(boolean expression) {
        // statements to execute if true
    }
    else if(boolean expression) {
        // statements to execute if true
    }
    else {
        // statements to execute if no other `if' was true
    }
\end{verbatim}
There are a few rules regarding \verb#else# and \verb#else if#:
\begin{itemize}
    \item There can be zero or one \verb#else# block for a given \verb#if#, and 
    it must come after any \verb#else if#s
    \item There can be zero to many \verb#else if#s for a given \verb#if# and 
    they must come before the (optional) \verb#else#
    \item Once any \verb#if# or \verb#else if# succeeds, none of the remaining 
    \verb#else if#s or the (optional) \verb#else# will be tested
\end{itemize}
For nested \verb#if# statements, an \verb#else# clause belongs to the innermost 
\verb#if# statement to which is might possibly belong (in other words, the 
closest preceeding \verb#if# that doesn't have an \verb#else#)

\subsubsection{switch Statements}
A \verb#switch# statement can emulate a nested \verb#if# by using \verb#break# 
statements within each \verb#case#. In general, \verb#break# statements are 
optional and their presence can alter the execution flow within a \verb#switch# 
statement. The basic format is:
\begin{verbatim}
    switch(expression) {
        case constant1: code block
        case constant2: code block
        default: code block
    }
\end{verbatim}
The expression must evaluate to a \verb#char#, \verb#byte#, \verb#short#, 
\verb#int# or \verb#enun# (note that boxing is permitted). In other words, 
anything that can be implicitly cast to an \verb#int#. Other types will fail 
compilation.  Each \verb#case# constant must evaluate to the same type as the 
expression (without losing precision) with the additional constraint that it 
must be a compile time constant. Note that using a final variable isn't a 
compile time constant unless it is declared and initialised in a single 
statement (eg \verb#final int a = 1#). Also note that a \verb#switch# can only 
check for equality and that it is illegal to have multiple \verb#case# labels 
using the same value

\subsubsection{Break and Fall-Through in switch Blocks}
\verb#case# constants are evaluated from the top down, and the first match is 
the execution entry point. If the \verb#break# keyword is encountered during 
the execution of a \verb#switch# statement, execution will immediately move out 
of the \verb#switch# block to the next statement after the \verb#switch#. If 
\verb#break# is omitted, the program will continue to execute the remaining 
\verb#case# blocks until either a \verb#break# is encountered or the 
\verb#switch# statement ends

\subsubsection{The default Case}
Note that the \verb#default# case doesn't have to appear at the end of the 
\verb#switch# and it acts just like any other \verb#case# in terms of 
fall-through

\subsection{Loops and Iterators}
There are three types of loops: \verb#while#, \verb#do# and \verb#for#. All 
allow the repetition of a block of code as long as some condition is true, or 
for a specific number of iterations

\subsubsection{Using while Loops}
The \verb#while# loop is useful for scenarios where you don't know how many 
times a block should repeat but you want to continue looping as long as some 
condition is true. The basic syntax is:
\begin{verbatim}
    while(boolean expression) {
        // statements to execute if condition is true
    }
\end{verbatim}
Note that any variables used in the while expression must be declared before 
the expression is evaulated (i.e. they cannot be declared within the while 
expression)

\subsubsection{Using do Loops}
The \verb#do# loop is similar to the \verb#while# loop, except that the 
expression is not evaluated until after the loop's code is executed. Therefore, 
the code in a \verb#do# loop is guaranteed to execute at least once. The basic 
syntax is:
\begin{verbatim}
    do {
        // statements to execute at least once
    } while(boolean expression);    // NOTE: semicolon is required
\end{verbatim}

\subsubsection{Using Basic for Loops}
The \verb#for# loop is especially useful for flow control when you know how 
many times the statements should be executed. There are four main parts to the 
structure:
\begin{itemize}
    \item Declaration and initialisation of variables
    \item The \verb#boolean# expression (conditional test)
    \item The iteration expression
    \item The loop body
\end{itemize}
The first three parts are separated by semicolons. The basic syntax is:
\begin{verbatim}
    for(/* initialisation*/ ; /* boolean condition */ ; /* iteration */) {
        // loop body
    }
\end{verbatim}
The initialisation part of the \verb#for# statement allows for the declaration 
and initialisation of zero or more variables of the same type separated by 
commas. Note that the scope of any newly declared variables ends with the scope 
of the \verb#for# loop. Note that there can be only one \verb#boolean# 
condition (unlike the initialisation and iteration sections). The iteration 
expression is executed after each execution of the loop body. Multiple 
iteration expressions of variables of the same type are allowed (separated by 
commas). Note that barring a forced exit (eg. \verb#break#, \verb#return# or 
\verb#System.exit()#), evaluating the iteration expression and then evaluating 
the conditional expression are always the last two things that happen in a 
\verb#for# loop

\subsubsection{Basic for Loop Issues}
\begin{itemize}
    \item None of the three sections of the \verb#for# declaration are required
    \item Variables declared in the \verb#for# loop statement are all local to 
    the \verb#for# statement
    \item The three expressions in the \verb#for# statement do not need to 
    operate on the same variables
\end{itemize}

\subsubsection{Using the Enhanced for Loop (for Arrays)}
The enhanced \verb#for# loop has the following syntax:
\begin{verbatim}
    for(declaration : expression) {
        // loop body
    }
\end{verbatim}
The two pieces of the enhanced \verb#for# loop are:
\begin{description}
    \item[declaration] The newly declared block variable of type compatible 
    with the elements of the array you are iterating over. This variable will 
    be available within the loop body and its value will be the same as the 
    current array element
    \item[expression] This must evaluate to the array (of any type) you want to 
    iterate over
\end{description}
The enhanced \verb#for# loop assumes that, barring an early exit from the loop 
(eg. \verb#break#, \verb#return# or \verb#System.exit()#), you will always loop 
through every element of the array

\subsection{Using break and continue}
The \verb#break# statement causes the program to stop execution of the 
innermost loop and start processing the next line of code after the block. The 
\verb#continue# statement causes only the current iteration of the innermost 
loop to cease and the next iteration of the same loop to start once the 
iteration expression has run if the condition of the loop is met. Note that 
\verb#continue# statements must be inside a loop, otherwise a compilation error 
will occur; \verb#break# statements must be used inside either a loop or a 
\verb#switch# statement. Note that this does not apply to labeled \verb#break# 
statements

\subsubsection{Labeled and Unlabeled Statements}
Both \verb#break# and \verb#continue# statements can be labeled or unlabeled.  
Although many statements can be labeled, it's most common to use labels with 
loop statements in conjunction with \verb#break# and \verb#continue#. A label 
statement must be placed just before the statement being labeled and consists 
of a valid identifier that ends with a colon. Labeled \verb#break# and 
\verb#continue# statements are only needed within nested loops to indicate 
which loop you want to break from or continue with the next iteration. A 
labeled \verb#break# statement will exit the labeled loop (as opposed to the 
innermost loop). A labeled \verb#continue# statement will cause execution to 
continue with the next iteration of the labeled loop. Note that labeled 
\verb#break# and \verb#continue# statements must be within a loop that has the 
same label name, otherwise compilation will fail

\subsection{Handling Exceptions}
Exception handling allows developers to detect errors easily without writing 
special code to test return values. It allows for exception-handling code to be 
kept separate from the exception-generating code and allows for the 
exception-handling code to deal with a range of possible exceptions

\subsubsection{Catching an Exception Using try and catch}
Exception handling works by transferring the execution of a program to an 
appropriate exception handler when an exception occurs. The \verb#try# keyword 
is used to define a block of code in which exceptions can occur. One or more 
\verb#catch# clauses match a specific exception (or classes of exceptions) to a 
block of code that handles it. If one or more \verb#catch# blocks are present, 
they must immediately follow the \verb#try# block. Additionally, the 
\verb#catch# blocks must all follow each other, without any other statements or 
blocks in between. Also, the ordering of the \verb#catch# blocks affects which 
\verb#catch# block is executed

\subsubsection{Using finally}
A \verb#finally# block encloses code that is always executed (unless a 
\verb#System.exit()# occurs) at some point after the \verb#try# block, whether 
an exception is thrown or not. Even if there is a \verb#return# statement 
within the \verb#try# block, the \verb#finally# block executes right after the 
return statement is encountered, and before the \verb#return# statement is 
executed. If the \verb#try# block executes without exceptions, the 
\verb#finally# block is executed immediately after the \verb#try# block 
completes. If an exception was thrown, the \verb#finally# block exeuctes 
immediately after the appropriate \verb#catch# block completes. Note that it is 
illegal to use a \verb#try# clause without either a \verb#catch# or a 
\verb#finally# clause. Any \verb#catch# clauses must immediately follow the 
\verb#try# block. Any \verb#finally# clause must immediately follow the last 
\verb#catch# clause (or it must immediately follow the \verb#try# block if 
there is no \verb#catch#). In summary, it is legal to omit either the 
\verb#catch# or the \verb#finally# clause, but not both

\subsubsection{Defining Exceptions}
Exceptions are always a subclass of \verb#java.lang.Exception#

\subsubsection{Exception Hierarchy}
All exception classes are subtypes of class \verb#java.lang.Exception#. This 
class derives from the class \verb#Throwable# (which derives from 
\verb#Object#). The hierarchy is as follows:
\begin{verbatim}
    Object
        - Throwable
            - Error
            - Exception
                - RuntimeException
\end{verbatim}
There are two ways to get information about an exception. The first is from the 
type of the exception itself. The second is by calling methods inherited from 
\verb#Throwable# such as \verb#printStackTrace()#. For the exam, it is not 
necessary to know any of the methods contained in the \verb#Throwable# classes, 
including \verb#Exception# and \verb#Error#. However, you do need to know that 
\verb#Exception#, \verb#Error#, \verb#RuntimeException# and \verb#Throwable# 
types can all be thrown using the keyword \verb#throw#, and all can be caught 
(although you will rarely catch anything other than \verb#Exception#s)

\subsubsection{Handling an Entire Class Hierarchy of Exceptions}
It is possible to catch more than one type of exception in a single 
\verb#catch# clause since if the type specified has subclasses, they will be 
caught too. While this is convernient, it means you have less information 
available when handling the exception (unless you test to determine which class 
the exception belongs to using \verb#instanceof# -- which is just as much work 
as writing multiple \verb#catch# blocks)

\subsubsection{Exception Matching}
When an exception is thrown, java will try to find (by looking at the available 
\verb#catch# clauses from the top down) a \verb#catch# clause for the exception 
type. If it does not find a \verb#catch# clause that matches, then the 
exception is propagated down the call stack. This process is called exception 
matching

\subsubsection{Exception Declaration and the Public Interface}
Just as a method must specify what type and how many arguments it accepts and 
what it returns, the exceptions that a method can throw must be declared 
(unless the excptions are subclasses of \verb#RuntimeException#). The list of 
thrown exceptions is part of a method's public interface. The \verb#throws# 
keyword is used as follows to list the exceptions that a method can throw:
\begin{verbatim}
    void myFunction() throws MyException1, MyException2 {}
\end{verbatim}
Each method must either handle all checked (non-runtime) exceptions by 
supplying a \verb#catch# clause or list each checked exception as a thrown 
exception. Note that exceptions of type \verb#RuntimeException# may be thrown 
from any method without being specified as part of the method's public 
interface (and a handler need not be present). Even if the method does declare 
that it throws a \verb#RuntimeException#, the calling method is under no 
obligation to handle or declare it. \verb#RuntimeException#, \verb#Error# and 
all of their subtypes are unchecked exceptions and therefore do no need to be 
specified or handled.

Objects of type \verb#Error# are not \verb#Exception# objects, although they do 
represent exceptional conditions. Both \verb#Exception# and \verb#Error# share 
a common superclass, \verb#Throwable#, thus both can be thrown using the 
\verb#throw# keyword. When an \verb#Error# or one of its subclasses is thrown, 
it's unchecked, meaning you are not required to catch or declare it

\subsubsection{Rethrowing the Same Exception}
Just as you can throw a new exception from a \verb#catch# clause, you can also 
rethrow the same exception you just caught. All other \verb#catch# clauses 
associated with the same \verb#try# are ignored, if a \verb#finally# block 
exists, it runs before the exception is thrown. Note that if you throw a 
checked exception (even if it is from within a \verb#catch# block), you must 
still declare the exception as being thrown

\subsection{Common Exceptions and Errors}
For the purposes of the exam, there are two broad categories of exceptions and 
errors:
\begin{description}
    \item[JVM exceptions] Exceptions or errors that are either exclusively or 
    most logically thrown by the JVM
    \item[Programmatic exceptions] Exceptions that are thrown explicitly by 
    application and/or API programmers
\end{description}

\subsubsection{JVM Thrown Exceptions}
\begin{description}
    \item[NullPointerException] Occurs when you attempt to access an object 
    using a reference variable with a current value of \verb#null#.  There's no 
    way for the compiler to detect these problems before runtime
    \item[StackOverflowException] Occurs when the stack becomes too large
\end{description}

\subsubsection{Summary of Exam's Exceptions and Errors}
\begin{center}
    \begin{tabular}{p{0.35\textwidth}p{0.4\textwidth}l}
        \textbf{Exception} & \textbf{Description} & \textbf{Typically Thrown 
        By} \\
        \hline
        \verb#IllegalArgumentException# & Thrown when a method receives an 
        argument formatted differently than the method expected & Programmer \\
        \hline
        \verb#IllegalStateException# & Thrown when the state of the environment 
        doesn't match the operation being attempted & Programmer \\
        \hline
        \verb#NumberFormatException# (extends \verb#IllegalArgumentException#) 
        & Thrown when a method that converts a String to a number receives a 
        String that it cannot convert & Programmer \\
        \hline
        \verb#AssertionError# & Thrown when an assertion fails & Programmer \\
        \hline
        \verb#ArrayIndexOutOfBoundsException# & Thrown when attempting to 
        access an array with an invalid index (negative or beyond the length of 
        the array) & JVM \\
        \hline
        \verb#NullPointerException# & Thrown when attempting to access an 
        object with a reference variable whose current value is \verb#null# & 
        JVM \\
        \hline
        \verb#ClassCastException# & Thrown when attempting to cast a reference 
        variable to an incompatible type & JVM \\
        \hline
        \verb#ExceptionInInitializerError# & Thrown when attempting to 
        initialise a static variable or initialisation block & JVM \\
        \hline
        \verb#StackOverflowError# & Thrown when the stack overflows & JVM \\
        \hline
        \verb#NoClassDefFoundError# & Thrown when the JVM can't find a class it 
        requires & JVM \\
    \end{tabular}
\end{center}

\subsection{Working with Assertions}
Added in Java in 1.4, assertions allow for testing assumptions during 
development without the expense (in both programmer time and execution 
overhead) of writing exception handlers for exceptions that you assume will 
never occur once the program is deployed

\subsubsection{Assertions Overview}
You always assert that a condition is \verb#true#. If it is not, an 
\verb#AssertionError# is thrown (that you should never handle). There are two 
assertion syntaxes:
\begin{verbatim}
    assert(boolean condition);
    assert(boolean condition): Value to be added to stack trace;
\end{verbatim}
Note that in the second format, anything which results in a value can be used 
(note that \verb#void# method calls are not legal)

\subsubsection{assert Keyword}
Since assertions were introduced in java 1.4, prior to this version, ``assert" 
was not a keyword.  It is possible to compile code using java 1.3 or earlier 
using ``assert" as an identifier using the \verb#-source 1.3# javac flag. Note 
that if doing this, a compiler warning will be displayed

\subsubsection{Enabling Assertions at Runtime}
You can enable assertions at runtime using either:
\begin{verbatim}
    java -ea com.unico.TestAssertions
    java -enableassertions com.unico.TestAssertions
\end{verbatim}

\subsubsection{Disabling Assertions at Runtime}
You can disable assertions at runtime using either:
\begin{verbatim}
    java -da com.unico.TestAssertions
    java -disableassertions com.unico.TestAssertions
\end{verbatim}

\subsubsection{Selective Enabling and Disabling Assertions}
The command line switches for assertions can be used in various ways by 
appending `:' where necessary:
\begin{description}
    \item[With no arguments (as above)] Enables or disables assertions in all 
    classes (except system classes). Note that no ``:" is needed in this case
    \item[With a package name] Enables or disables assertions in the package 
    specified, and any packages below this package in the same directory 
    hierarchy. The package name should be followed by \ldots\ when used in this 
    fashion
    \item[With a class name] Enables or disables assertions in the class 
    specified
\end{description}
Some examples of selectively enabling and disabling assertions are:
\begin{center}
    \begin{tabular}{lp{.6\textwidth}}
        \textbf{Command Line} & \textbf{Meaning} \\
        \hline
        \verb#java -ea# & Enable assertions \\
        \hline
        \verb#java -da# & Disable assertions \\
        \hline
        \verb#java -eacom.foo.Bar# & Enable assertions in class com.foo.Bar \\
        \hline
        \verb#java -eacom.foo...# & Enable assertions in package com.foo and 
        any of its subpackages \\
        \hline
        \verb#java -ea -dsa# & Enable assertions but disable assertions in 
        system classes \\
        \hline
        \verb#java -ea -dacom.foo...# & Enable assertions but disable 
        assertions in package com.foo and any of its subpackages \\
    \end{tabular}
\end{center}

\subsubsection{Using Assertions Appropriately}
AssertionErrors should never be handled. To discourage you from handling them, 
\verb#AssertionError# doesn't provide access to the object that generated it.  
Some notes on the usage of assertions:
\begin{itemize}
    \item Don't use assertions to validate arguments to a public method, since 
    they are not guaranteed to run at all. Throw an 
    \verb#IllegalArgumentException# instead
    \item Don't use assertions to validate command-line arguments (special case 
    of above)
    \item Use assertions to validate arguments to private methods, since you 
    often want to ensure assumptions in your code are valid
    \item Use assertions (even in public methods) to check for cases which you 
    know should never happen (for example, by using \verb#assert false;# in the 
    default \verb#case# of a switch which should never match)
    \item Assert statements should never cause side effects since they are not 
    guaranteed to run
\end{itemize}
