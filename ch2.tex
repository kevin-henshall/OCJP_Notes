\section{Chapter 2 -- Object Orientation}
\subsection{Encapsulation}
The ability to make changes in implementation code without breaking the code of 
users of the code is a key benefit of encapsulation. You want to hide 
implementation details behind a public programming interface

\subsection{Inheritance, Is-A, Has-A}
You can create inheritance relationships by extanding a class. The two most 
common reasons to use inheritance are:
\begin{itemize}
    \item To promote code reuse
    \item To promote polymorphism
\end{itemize}

\subsection{Object Orientation}
The methods you can call on a reference variable are totally dependent on the 
declared type of the variable, regardless of what the type of the referenced 
object

\subsection{Is-A}
The concept os Is-A is based on class inheritance or interface implementation.  
You express the Is-A relationship through the keywords \verb#extends# for class 
inheritance, and \verb#implements# for interface implementation. If the 
expression \verb#Foo instanceof Bar# is true, then class Foo Is-A Bar, even if 
Foo does not directly extend Bar, but instead extends some other class that is 
a subclass of Bar

\subsection{Has-A}
Has-A relationships are based on usage, rather than inheritance. In other 
words, class A Has-A B if code in class A has a reference to an instance of 
class B

\subsection{Polymorphism}
The obly way to access an object is through a reference variable, and there are 
a few key things to remember about references:
\begin{itemize}
    \item A reference variable can be of only one type and once declared, that 
    type can never be changed (although the object it refers to can change)
    \item A reference is a variable, so it can be reassigned to other objects 
    (unless the reference is declared \verb#final#
    \item A reference variable's type determines the methods that can be 
    invoked on the object the variable is referencing
    \item A reference variable can refer to any object of the same type as the 
    declared reference, or a subtype of the declared type
    \item A reference variable can be declared as a class type or an interface 
    type. If the variable is declared as an interface type, it can reference 
    any object of any class that implements the interface
\end{itemize}
Remember that method invocations allowed by the compiler are based solely on 
the declared type of the reference, regardless of the object type. However, the 
compiler always knows that you can invoke the methods of class Object on any 
object, so those are safe to call regardless of the reference -- class or 
interface -- used to refer to the object. At runtime, the JVM knows what the 
object really is and it can determine which method will be invoked in cases of 
overridden methods.

Polymorphic method invocations apply only to instance methods. You can refer to 
an object with a more general reference variable type (a superclass or 
interface), but at runtime, the only things that are dynamically selected based 
on the actual object (rather than the reference type) are instance methods -- 
not static variables or variables. Only overridden instance methods are 
dynamically invoked based on the object's real type.

\subsection{Overriding Methods}
Any time you have a class that inherits a method from a superclass, you have 
the opportunity to override the method, unless the method is marked 
\verb#final#. The rules for overriding a method are:
\begin{itemize}
    \item The argument list must exactly match that of the overridden method
    \item The return type must be the same as, or a subtype or, the return type 
    of the overridden method
    \item The access level can not be more restrictive than that of the 
    overridden method
    \item The access level can be less restrictive that that of the overridden 
    method
    \item Instance methods can be overridden only if they are inherited
    \item The overriding method can throw any unchecked (runtime) exceptions, 
    regardless of whether the overridden method declared the exception
    \item The overriding method can not throw checked exceptions that are new 
    or broader than those declared by the overridden method
    \item The overriding method can throw narrower or fewer exceptions than the 
    overridden method
    \item Methods marked \verb#final# cannot be overridden
    \item Methods marked \verb#static# cannot be overridden
    \item If a method cannot be inherited, it cannot be overridden
\end{itemize}

\subsection{Invoking a Superclass version of an Overridden Method}
An overridden superclass version of a method can be invoked by using 
\verb#super#. Note that \verb#super# can only be used to invoke an overridden 
instance methods (remember that \verb#static# methods are not inherited). Note 
that if a method is overridden but you use a supertyp reference to refer to the 
subtype object with the overriding method, the compile assumes you're calling 
the supertype version of the method. If the supertype version declares a 
checked exception, but the overriding subtype method does not, the compile 
still thinks you're calling calling a method that declares an exception

\subsection{Overloaded Methods}
Overloaded methods let you reuse teh same method name in a class but with 
different arguments (and optionally, a different return type). The rules are:
\begin{itemize}
    \item The argument list must change
    \item The return type can change
    \item The access modifier can change
    \item New or broader checked exceptions can be declared
    \item A method can be overloaded in the same class or in a subclass (in the 
    case where the superclass version is inherited
\end{itemize}

\subsection{Deciding Which Overridden and Overloaded Method to Invoke}
\begin{itemize}
    \item The reference type (not the object type) of the argument determintes 
    which overloaded method is invoked at compile time
    \item The object type (not the reference type) determines which overridden 
    method is invoked at runtime
\end{itemize}

\subsection{Differences between Overridden and Overloaded Methods}
\begin{center}
\begin{tabular}{lp{6cm}p{6cm}}
    & \textbf{Overloaded Method} & \textbf{Overridden Method} \\
    \hline
    Argument list & Must change & Must not change \\
    \hline
    Return type & Can change & Can't change -- except for covariant returns \\
    \hline
    Exceptions & Can change & Can reduce or eliminate. Can't throw new or 
    broader checked exceptions \\
    \hline
    Access & Can change & Can't be more restrictive \\
    \hline
    Invocatoion & Reference type determines which overloaded version (based on 
    declared argument types) is invoked. Happens at compile time & Object type 
    determines which method is selected. Happens at runtime \\
\end{tabular}
\end{center}

\subsection{Reference Variable Casting}
A downcase is a cast down the inheritance tree to a more specific type. Note 
that the compiler trusts us when making a downcast -- even if it is wrong, 
since the compiler can only verify that the two types are in the same 
inheritance tree. However, if the compiler knows with certainty that the case 
could not possibly work, compilation will fail.

Unlike downcasting, upcasting (casting up the inheritance tree to a more 
general type) works implicitly since when upcasting, you're restricting the 
methods which can be invoked.

Note that a concrete class can be declared to implement an interface without 
impelemnting the methods of the interface. If any supertype in its inheritance 
tree has already provided concrete (i.e. non-abstract) method impelementations 
for the interface, then regardless of whether the superclass declares that it
implements the interface, the subclass is under no obligation to re-implement 
(override) those methods

\subsection{Implementing an Interface}
WHen you implement an interface, you're agreeing to adhere to the contract 
defined in the interface. Implementation classes must adhere to the same rules 
for method implementation as a class extending an \verb#abstract# class. In
order to be a legal implementation class, a nonabstract implementation class 
must do the following:
\begin{itemize}
    \item Provide concrete implements for all methods from the declared 
    interface
    \item Follow all the rules for legal overrides
    \item Declare no checked exceptoins on impelementation methods other than 
    those declared by teh interface method, or subclasses of those declared by 
    the interface method
    \item Maintain the signature of the interface method and maintain the 
    return type (or a subtype). Note that it does not need to declared the 
    exceptions declared in the interface method declaration
\end{itemize}
Note that if the implementation class is abstract, it does not need to provide 
method implementations but the first concrete class must. Also note:
\begin{itemize}
    \item A class can implement more than one interface
    \item An interface can extend more than one interface, but can never 
    implement anything
    \item A class can both extend another class and impelment an interface, 
    however the \verb#extends# statement must be first
\end{itemize}

\subsection{Return Types on Overloaded Methods}
Overloaded method are not subject to the restrictions of overriding, which 
means you can declare any return type. What you can't do is change only the 
return type.

\subsection{Return Types on Overridden Methods}
Method overrides must match the declaration of the overridden method exactly.  
However, the return type may be a subtype of that declared in the overridden 
method

\subsection{Returning a Value}
There are six rules for returning a value:
\begin{itemize}
    \item You can return \verb#null# from a method with an object reference 
    return type
    \item An array is a legal return type
    \item From a method with a primitive return type, you can return any value 
    or variable that can be implicitly converted to the declared return type
    \item From a method with a primitive return type, you can return any value 
    or variable that can be explicitly cast to the declared return type
    \item You must not return anything from a method with a void return type.  
    However, \verb#return# can be used to return from a method without 
    returning a value
    \item From a method with an object reference return type, you can return 
    any object type that can be implicitly cast to the declared return type 
    (including interfaces)
\end{itemize}

\subsection{Constructors}
Every class, including abstract classes, must have a constructor. However, just 
because a class must have one, doesn't mean that you need to type it. Note that 
constructors:
\begin{itemize}
    \item Have no return type
    \item Must have a name which matches the class name
\end{itemize}

\subsection{Rules for Constructors}
The following rules apply for constructors:
\begin{itemize}
    \item Constructors can use any access modifier. A private constructor means 
    only code within the class itself can instantiate an object of that type, 
    so if you want to allow an instance of the class to be used, there must be 
    a static method or variable that allows access to an instance created from 
    within the class
    \item The constructor name must match teh name of the class
    \item Constructors must not have a return type
    \item It's legal to have a method with the same name as the class
    \item If you don't type a constructor, a default constructor will be 
    generated by the compiler
    \item The default constructor is always a no-arg constructor
    \item If you want a no-arg constructor and you've type any other 
    constructors into the class, the compiler won't provide the no-arg (or any 
    other) constructor for you
    \item Every constructor has, as its first statement, either a class to an 
    overloaded constructor (\verb#this()#) or a class to the superclass 
    constructor (\verb#super()#), however, this call can be inserted by the
    compiler
    \item If you do create a constructor and do not provide the required call 
    to \verb#this()# or \verb#super()#, the compiler will insert a no-arg call 
    to \verb#super()# as the very first statement in the constructor
    \item A call to \verb#super()# can be either a no-arg call or can include 
    arguments passed to the super constructor
    \item A no-arg constructor is no necessarily the default constructor, 
    although the default constructor is always a no-arg constructor
    \item You cannot make a call to an instance method or access an instance 
    variable until after the super constructor completes
    \item Only static variables and methods can be accessed as part of the call 
    to \verb#super()# or \verb#this()#. That is, only static variables and 
    methods can be passed as arguments to \verb#super()# or \verb#this()#
    \item Abstract classes have constructors and those constructors are always 
    called when a concrete subclass is instantiated
    \item Interfaces do not have constructors -- they are not part of an 
    object's inheritance tree
    \item The only way a constructor can be invoked is from within another 
    constructor. Note that this is different from calling \verb#new# to 
    instantiate an instance of a class
\end{itemize}

\subsection{Determining Whether a Default Constructor Will Be Created}
If no constructor is typed for a class, a default constructor will be created.  
The default constructor:
\begin{itemize}
    \item Has the same access modifier as the class
    \item Has no arguments
    \item Includes a no-arg call to the super constructor (\verb#super()#)
\end{itemize}
Note that if the superclass does not have a no-arg constructor, then subclasses 
cannot use the default constructor supplied by the compiler.

\subsection{Overloaded Constructors}
Overloading a constructor means typing in multiple versions of the constructor, 
each having a different argument list. Overloading a constructor is typically 
used to provide alternate ways for clients to instantiate objects of the class.  
Note that the compiler will not insert a call to \verb#super()# into any 
constructor which already has a call to \verb#this()#

\subsection{Static Variables and Methods}
Static members are used to address the utility-method-always-runs-the-same and 
the keep-a-running-total-of-instances scenarios. Variables and methods marked 
\verb#static# belong to the class, rather than any particular instance. In 
fact, you can use a \verb#static# method or variable without having any 
instances of the class at all. You need only have the class available to be 
able to invoke a \verb#static# method or access a \verb#static# variable. Note 
that static variables will be shared by all instances of the class. Note that 
\verb#static# variables get the same default values instance variables get.

Note that a \verb#static# method can't access a nonstatic (instance) variable 
or method of its own class. Making the JVM \verb#main()# method a static method 
means the JVM doesn't have to create an instance of the class just to start 
running code

\subsection{Accessing Static Methods and Variables}
To access a static method or variable, use the dot operator on the class name.  
However, you can also use an object reference to access a static member. Note 
that this is simply a syntax trick and the runtime object is not inspected.  
Note that static methods cannot be overridden -- this doesn't mean they cannot 
be redefined in a subclass, but redefining and overriding are not the same 
thing.

\subsection{Coupling and Cohesion}
Coupling and Cohesion related to the quality of an OO design. In general, good 
OO design calls for loose coupling and shuns tight coupling, and good OO design 
calls for high cohesion and shuns low cohesion.

\subsection{Coupling}
Coupling is the degree to which one class knows about another class. If A knows 
more than it should about the way in which B is implemented, then A and B are 
tightly coupled

\subsection{Cohesion}
Cohesion is the degree to which each class has a single well focused purpose
