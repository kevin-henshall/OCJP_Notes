\section{Chapter 10 -- Development}
\subsection{Using the java and javac Commands}
\subsubsection{Compiling with javac}
The \verb#javac# command is used to invoke java's compiler. For the exam, you 
will need to understand the \verb#-classpath# and \verb#-d# options. The 
structured overview for \verb#javac# is: \verb#javac [options] [source files]#.
There are additional command-line options called \verb#@argfiles#, bu you will 
not need to study them for the exam. Both the \verb#[options]# and the 
\verb#[source files]# are optional parts of the command, and both allow 
multiple entries. Whenever you specify multiple options and/or files they 
should be separated by spaces

\subsubsection{Compiling with -d}
By default, the compiler puts a \verb#.class# file in the same directory as the 
\verb#.java# source file. The \verb#-d# option lets you tell compiler in which 
directory to put the \verb#.class# file(s) it generates ('d' is for 
destination). For example:
\begin{verbatim}
    package com.wickedlysmart;
    public class MyClass {}

    myProject/
        source/
            com/
                wichedlysmart/
                    MyClass.java
        classes/
\end{verbatim}
If the following command is run from the \verb#source# directory, it would 
compile \verb#MyClass.java# and put the resulting \verb#MyClass.class# into the 
\verb#classes/com/wickedlysmart/# directory:
\begin{verbatim}
    javac -d ../classes com/wickedlysmart/MyClass.java
\end{verbatim}
Since \verb#MyClass.java# is in a package, the compiler knew to put the 
resulting \verb#.class# file into the \verb#classes/com/wickedlysmart/# 
directory. The \verb#javac# command can create some of the directories it 
needs, however, the class file destination (specified with \verb#-d#) must 
exist, otherwise compilation will fail

\subsubsection{Launching Applications with java}
The \verb#java# command is used to invoke the jvm. When running the \verb#java# 
command, you will need to understand the \verb#-classpath# (and its twin 
\verb#-cp#) and \verb#-D# options. In iddition, it is important to understand 
the structure of this command: \verb#java [options] class [args]#. The 
\verb#[options]# and \verb#[args]# parts are optional, and they can both have 
multiple values. You must specify exactly one class file to execute and the 
\verb#java# command assumes you are invoking a \verb#.class# file, so you do 
not need to specify the \verb#.class# extension on the command line

\subsubsection{Using System Properties}
As of java 5, the \verb#java.util.Properties# class can be used to access a 
system's persistent information as well as your own properties. Note that to 
obtain the system properties, we use \verb#System.getProperties()#, then 
\verb#getProperty()# and \verb#setProperty()# as required. There is also a 
\verb#list()# method:
\begin{verbatim}
    import java.util.*;
    public class TestProperties {
        public static void main(String[] args) {
            Properties properties = System.getProperties();
            p.setProperty("myProperty", "myValue");
            p.list(System.out);
        }
    }
\end{verbatim}
This example can be compiled and invoked as \verb#java -DcmdProperty=cmdValue TestProperties#. Note that if the value provided to a \verb#-D# option contains 
a space, it should be enclosed in double quotes. Also note that there is no 
space between the \verb#-D# and the property name.

The \verb#getProperty()# method is used to retrieve a single property. It can 
be invoked with a single argument (a \verb#String# that represents the name (or 
key)), or it can be invoked with two arguments (a \verb#String# that represents 
the name (or key), and a default \verb#String# value to return if the property 
does not exist). In both cases, \verb#getProperty()# returns a \verb#String#

\subsubsection{Handling Command-Line Arguments}
Like all arrays, the \verb#String[] args# array passed to \verb#main()# is zero 
based. Arguments on the command line directly follow the class name. The first 
argument is assigned to \verb#args[0]#, the second to \verb#args[1]# and so on

\subsubsection{Searching for Other Classes}
Both \verb#java# and \verb#javac# use the same basic search algorithm:
\begin{enumerate}
    \item They both have the same list of places (directories) they search for 
    classes
    \item They both search through this list of directories in the same order
    \item As soon as they find the class they are looking for, they stop 
    searching for that class. In the case that the search list contains 
    duplicates, the first file found will be used
    \item The first place they look is in the directories that contain the 
    classes which are shipped with J2SE
    \item The second place they look is in the directories defined by 
    classpaths
\end{enumerate}
Classpaths should be thought of as `class search paths'. They are lists of 
directories in which classes might be found. There are two places where 
classpaths can be declared:
\begin{itemize}
    \item A classpath can be declared as an OS environment variable. The 
    classpath declared here is used by default, whenever \verb#java# or 
    \verb#javac# is invoked
    \item A classpath can be declared as a command-line option for either 
    \verb#java# or \verb#javac#. Classpaths declared as command-line options 
    override the classpath declared as an environment variable, but they 
    persist only for the length of invocation
\end{itemize}

\subsubsection{Declaring and Using Classpaths}
Classpaths consist of a variable number of directory locations, separated by a 
colon (:). Note that when you specify a subdirectory, you are not specifying 
its parent directories. When searching for class files, the \verb#java# and 
\verb#javac# commands do not search the current directory by default. However, 
\verb#javac# does search the current directory for java files when compiling.  
Note that classpaths are searched from left to right and so order is important. 
Finally, note that when used with the \verb#java# command, you can abbreviate 
\verb#-classpath# to \verb#-cp#. While most \verb#javac# implementtations 
support the same abbreviation, there are no guarantees

\subsubsection{Packages and Searching}
If a class (\verb#MyClass#) is within a package (\verb#com.foo#), then the 
fully qualified name of the class is \verb#com.foo.Myclass#. Once a class is in 
a package, the package part of its fully qualfied name is atomic -- it can 
never be divided. You cannot split it up on the command line, and you cannot 
split it up in an \verb#import# statement.

The \verb#import# statement is like an alias for a class's fully qualified 
name. You define the fully qualified name for the class with an \verb#import# 
statement (or with a wildcard in an \verb#import# statement of the package).  
Once you've defined teh fully aualified name, you can use the alias in your 
code -- just remember that the alias is referring back to the fully qualified 
name.

Whe you are searching for a class using its fully qualified name, that fully 
qualified name relates closely to a specific directory structure. In order to 
find a class in a package, you need to have a directory in your classpath that 
has the package's leftmost entry (the package's root) as a subdirectory.
\begin{verbatim}
    import com.wickedlysmart.Utils;
    class TestClass {
        void doStruff() {
            Utils u = new Utils();                                          // alias
            u.doX("arg1", "arg2");
            com.wickedlysmart.Date d = new new com.wickedlysmart.Date();    // full name
            d.getMonth("Oct");
        }
    }
\end{verbatim}
In this case, we are using two classes from hte package 
\verb#com.wickedlysmart#. We imported the fully qualified name for the 
\verb#Utils# class, and we did not for the \verb#Date# class. The only 
difference is that becuase we listed \verb#Utils# in an \verb#import# 
statement, we do not have to type its fully qualified name within the class. In 
both cases, the package is \verb#com.wickedlysmart#. When compiling or running 
\verb#TestClass#, the classpath will need to include a directory with the 
following attributes:
\begin{itemize}
    \item A subdirectory named \verb#com# (we will call this the `package root' 
    directory)
    \item A subdirectory within \verb#com# named \verb#wickedlysmart#
    \item Tow files within \verb#wickedlysmart# named \verb#Utils.class# and 
    \verb#Date.class#
    \item Finally, the directory that has all of these attributes needs to be 
    accessible (via a classpath) in one of two ways:
    \begin{itemize}
        \item The path to the directory must be absolute, or
        \item The path to the directory has be correct valid to the current 
        directory
    \end{itemize}
\end{itemize}

\subsubsection{Absolute and Relative Paths}
A classpath is a collection of one or more paths. Each path in a classpath is 
either an absolute or relative path. Note that with absolute paths, it does not 
matter what the current directory is since an absolute path is always the same.
Note that the current directory is only searched if \verb#.# is in the 
classpath, or if it is covered by an absolute or relative path

\subsection{Jar Files}
\subsubsection{Jar Files and Searching}
Once a jar is created, it can be moved and copied to different machines, and 
all of the classes within it can be accessed, via classpaths, by \verb#java# 
and \verb#javac#, without unjaring the jar file. Some rules concerning the 
structure of jar files are:
\begin{itemize}
    \item The \verb#jar# command creates the \verb#META-INF# directory 
    automatically
    \item The \verb#jar# command creates the \verb#MANIFEST.MF# file 
    automatically
    \item The \verb#jar# command will not place any of your files within 
    \verb#META-INF#
    \item The directory hierarchy is maintained within the jar file
    \item \verb#java# and \verb#javac# will use the jar file like a normal 
    directory tree
\end{itemize}
Finding a jar file using a classpath is similar to finding a package file in a 
classpath. The difference is that when you specify a path for a jar file, you 
must include the name of the jar file at the end of the path. Remember that 
when using classpaths, the last directory in the path must be the super 
directory of the root directory of the package.

When you use an \verb#import# statement, you are declaring only one package.  
When you say \verb#import java.util.*#, you are saying, `use the short name for 
all of the classes in the \verb#java.util# package'. You are not getting the 
\verb#java.util.jar# classes or the \verb#java.util.regex# packages. Those 
packages are completely independent of each other; the only thing they share is 
the same `root' directory, but they are not in the same packages

\subsubsection{Using jre/lib/ext with Jar Files}
If you put jar files into the \verb#$JAVA_HOME/jre/lib/ext/# directory, 
\verb#java# and \verb#javac# can find them and use the class files thay 
contain. You do not have to mention these subdirectories in a classpath 
statement -- searching this directory is a function that is built-in to java

\subsection{Using Static Imports}
Ultimately, the only value \verb#import# statements have is that they save 
typing and they can make your code easier to read. In java 5, the \verb#import# 
statement was enhanced to provide even greater keystroke-reduction capabilities 
-- although som would argue that this comes at the expense of readability.

Some rules for using static imports:
\begin{itemize}
    \item Even though the feature is referred to as `static imports', the code 
    must read \verb#import static#, followed by the fully qualified name of the 
    \verb#static# member you want to import
    \item Be aware of ambiguously named \verb#static# members. For instance, if 
    you have a \verb#static# import for both the \verb#Integer# and \verb#Long# 
    classes, referring to \verb#MAX_VALUE# will cause a compilation error, 
    since both \verb#Integer# and \verb#Long# have a \verb#MAX_VALUE# constant
    \item You can use a static imports on \verb#static# object references, 
    constants (remember they are \verb#static# and \verb#final#), and 
    \verb#static# methods
\end{itemize}
