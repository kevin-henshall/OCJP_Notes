\section{Chapter 6 -- Strings, IO, Formatting and Parsing}
\subsection{String, StringBuffer and StringBuilder}
\subsubsection{The String class}
Once a \verb#String# object is created, it can never be changed

\subsubsection{Important Facts About Strings and Memory}
To make java more efficient, the JVM sets aside a special area of memory called 
the ``String constant pool". When the compiler encounters a String literal, it 
checks the pool to see if an identical String already exists. If a match is 
found, the reference to the new literal is directed to the existing String, and 
no new String literal object is created (the existing String simply has an 
additional reference)

\subsubsection{Creating New Strings}
There are a few ways to create Strings:
\begin{itemize}
    \item \verb#String s = "abc";# // ``abc'' will be placed in the String pool 
    and \verb#s# will refer to it
    \item \verb#String s = new String("abc");# // since `new' is used, a new 
    String object will be created in normal (non-pool) memory and \verb#s# will 
    refer to it. Also, ``abc" will be placed in the String pool
\end{itemize}

\subsubsection{Important Methods in the String Class}
\begin{itemize}
    \item \verb#charAt()# -- Returns the character located at the specified 
    0-index
    \item \verb#concat()# -- Appends one String to the end of another (`+' also 
    works)
    \item \verb#equalsIgnoreCase()# -- Determines the equality of two Strings, 
    ignoring case
    \item \verb#length()# -- Returns the number of characters in the String.  
    Note that Strings have a \verb#length()# method whereas Arrays have a 
    \verb#length# attribute
    \item \verb#replace()# -- Replace all occurrences of a character with a new 
    character
    \item \verb#substring()# -- Returns a part of a String. Note that the first 
    index is 0-based and the second 1-based
    \item \verb#toLowerCase()# -- Returns a String with uppercase characters 
    converted
    \item \verb#toUpperCase()# -- Returns a String with lowercase characters 
    converted
    \item \verb#toString()# -- Returns the value of a String
    \item \verb#trim()# -- Removes leading and trailing whitespace from a 
    String
\end{itemize}

\subsubsection{The StringBuffer and StringBuilder Classes}
The \verb#java.lang.StringBuffer# and \verb#java.lang.StringBuilder# classes 
should be used when you need to make lots of modifications to strings of 
characters. They can be modified without leaving discarded Strings in memory as 
is the case when using the \verb#String# class

\subsubsection{StringBuffer versus StringBuilder}
The \verb#StringBuilder# class was added in java 5 and has the same API as the 
\verb#StringBuffer# class except that \verb#StringBuilder# is not thread safe.  
In other words, its methods are not synchronized

\subsubsection{Important Methods in the StringBuffer and StringBuilder Classes}
\begin{itemize}
    \item \verb#append(String s)# -- Takes many argument types though most 
    common is \verb#String# and appends the argument to the invoking object
    \item \verb#delete(int start, int end)# -- Deletes the substring between 
    the 0-based start index and the 1-based end index
    \item \verb#insert(int offset, String s)# -- Takes many arguments types for 
    the second argument, though most common is \verb#String# and inserts the 
    argument at the given 0-based offset in the invoking object
    \item \verb#reverse()# -- Reverses the characters of the invoking object
    \item \verb#toString()# -- Returns the value of the invoking object
    \item \verb#equals()# -- Note that neither \verb#StringBuilder# nor
    \verb#StringBuffer# override \verb#Object#'s \verb#equals()# method
\end{itemize}

\subsection{File Navigation and IO}
Here is a summary of the IO classes for the exam:
\begin{itemize}
    \item \verb#File# -- An abstract representation of file and directory 
    pathnames
    \item \verb#FileReader# -- Used to read character files. Usually wrapped by 
    higher-level Reader objects such as BufferedReaders
    \item \verb#BufferedReader# -- Used to make lower-level Reader classes 
    (like FileReader) more efficient and easier to use
    \item \verb#FileWriter# -- Used to write character files. Usually wrapped 
    by higher-level Writer objects such as BufferedWriters or PrintWriters
    \item \verb#BufferedWriter# -- Used to make lower-level Writer classes 
    (like FileWriter) more efficient and easier to use
    \item \verb#PrintWriter# -- Enhanced significantly in java 5 to make it 
    easier to use rather than wrapping a Writer in a FileWriter and 
    BufferedWriter. More of a convenience class
    \item \verb#Console# -- Introduced in java 6, it allows reading from and 
    writing to the console
\end{itemize}

\subsubsection{Flushing Output and Closing the File}
The \verb#flush()# method guarantees that data is actually written to a file.  
Once you have finished either reading or writing a file, you should close it 
using the \verb#close()# method

\subsubsection{Combining IO Classes}
\begin{center}
\begin{tabular}{llll}
    \textbf{java.io Class} & \textbf{Extends} & \textbf{Key Constructor 
    Arguments} & \textbf{Key Methods} \\
    \hline
    File & Object & String & \verb#createNewFile()#, \verb#exists()#, \\
    & & File, String & \verb#isDirectory()#, \verb#isFile()#, \\
    & & String, String & \verb#list()#, \verb#mkdir()#, \\
    & & & \verb#renameTo()#, \verb#delete()# \\
    \hline
    FileReader & Reader & File & \verb#read()#, \\
    & & String & \verb#close()# \\
    \hline
    BufferedReader & Reader & Reader & \verb#read()#, \verb#readLine()#, \\
    & & & \verb#close()# \\
    \hline
    FileWriter & Writer & File & \verb#write()#, \verb#flush()#, \\
    & & String & \verb#close()# \\
    \hline
    BufferedWriter & Writer & Writer & \verb#write()#, \verb#newLine()#, \\
    & & & \verb#flush()#, \verb#close()# \\
    \hline
    PrintWriter & Writer & File (as of java 5) & \verb#write()#, \\
    & & String (as of java 5) & \verb#format()#, \verb#printf()#, \\
    & & OutputStream & \verb#print()#, \verb#println()#, \\
    & & & \verb#flush()#, \verb#close()# \\
\end{tabular}
\end{center}

\subsubsection{Working with Files and Directories}
The statement \verb#File f = new File("foo");# always creates a \verb#File# and 
does one of two things:
\begin{itemize}
    \item If ``foo" does not exist, no file is created
    \item If ``foo" does exist, \verb#f# refers to the existing file
\end{itemize}
Note that the statement will never create an actual file. There are two ways to 
create a file:
\begin{itemize}
    \item Invoke \verb#createNewFile()# on a File object
    \item Create a Writer or Stream. Note that if creating a file within a 
    directory, then the directory must exist, otherwise a 
    \verb#java.io.IOException# is thrown
\end{itemize}
Notes on \verb#delete()# and \verb#renameTo()#:
\begin{itemize}
    \item \verb#delete()# -- Directories can only be deleted if they are empty
    \item \verb#renameTo()# -- Must be invoked using an existing \verb#File# 
    object and a valid new \verb#File# object as the argument (a 
    \verb#NullPointerException# will occur if the argument is null). Note that 
    non-empty directories can be renamed
\end{itemize}

\subsubsection{The java.io.Console Class}
If you're running java SE6 from the command line, you'll typically have access 
to a console object, to which you can get a reference by invoking 
\verb#System.console()#. The \verb#Console# class makes it easy to read input 
from the command line, both echoed and nonechoed (such as passwords), and makes 
it easy to write formatted output to the command line. On the input side, the 
two methods to understand are:
\begin{itemize}
    \item \verb#readLine()# -- Returns a String containing the input
    \item \verb#readPassword()# -- Returns a char array (to prevent the 
    password from being stored in a String pool)
\end{itemize}

\subsection{Serialisation}
Serialisation lets you simply say ``save this object and all of its instance 
variables unless I've explicitly marked an instance variable as tranisient"

\subsubsection{Working with ObjectOutputStream and ObjectInputStream}
The magic of basic serialisation uses two methods:
\begin{itemize}
    \item \verb#ObjectOutputStream.writeObject()# -- Serialise and write
    \item \verb#ObjectInputStream.readObject()# Read and deserialise
\end{itemize}
The \verb#java.io.ObjectOutputStream# and \verb#java.io.ObjectInputStream# 
classes are usually wrapped around lower level classes such as 
\verb#java.io.FileOutputStream# and \verb#java.io.FileInputStream#.

When you serialise an object, java serialisation takes care of saving that 
object's entire `object graph'. This means that a deep copy of everything 
saved needs to be restored. Remember that you need to make a conscious choice 
to create objects that are serialisable, by implementing the (marker) 
\verb#java.io.Serializable# interface

\subsubsection{Using writeObject and readObject}
\verb#writeObejct()# and \verb#readObject()# can be used to save and restore 
the values of transient variables. These are private methods which you can 
implement in your class that, if present, will be invoked automatically during 
serialisation and deserialisation. It's almost as if these methods were defined 
in the \verb#Serializable# interface, but they aren't. They are part of a 
special callback contract the serialisation system offers. The methods must 
have signatures matching:
\begin{verbatim}
    private void readObject(ObjectInputStream is)
        throws IOException, ClassNotFoundException {}
    private void writeObject(ObjectOutputStream os)
        throws IOException {}
\end{verbatim}
Like most IO related methods, \verb#writeObject()# and \verb#readObject()# can 
throw exceptions. When invoking \verb#defaultWriteObject()# 
(\verb#defaultReadObject()#) from within \verb#writeObject()# 
(\verb#readObject()#), you're telling the JVM to perform the normal 
serialisation process, however, usually some custom writing and reading is 
performed as well. Note that any custom components written need to be read back 
in the same order they were written.

Not all classes are serialisable (i.e. Object is not serialisable) as there are 
some objects which are runtime specific

\subsubsection{How Inheritance Affects Serialisation}
If a superclass is Serializable, then according to normal java interface rules, 
all subclasses of that class automatically implement Serializable implicitly.  
In other words, a subclass of a class marked Serializable passes the IS-A test 
for Serializable, and can be saved without having to explicitly mark the
    subclass as Serializable. You simply cannot tell whether a class is or is 
    not Serializable unless you can see the class inheritance tree to see if 
    any superclasses implement Serializable. If the class does not explicitly 
    extend any other class, and does not implement Serializable, then you know 
    for certain that the class is not Serializable, because Object does not 
        implement Serializable

\subsubsection{Object Construction using new}
\begin{itemize}
    \item All instance variables are assigned default values
    \item The constructor is invoked which immediately invokes the superclass 
    constructor (or another overloaded constructor, until one of the overloaded 
    constructors invokes the superclass constructor)
    \item All superclass constructors complete
    \item Instance variables that are initialised as part of their declaration 
    are assigned their initial values (as opposed to the default values they're 
    given prior to the superclass constructor completing)
    \item The constructor completes
\end{itemize}

\subsubsection{Object Construction when Deserialising}
When an instance of a serializable class is deserialised, the constructor does 
not run, and instance variables are not given their initially assigned values.  
If there are variables marked \verb#transient#, they will not be restored to 
their original state (unless you implement \verb#readObeject()#), but will 
instead be given the default value for that data type.

If class B is serialisable, but it's superclass A is not, then any instance 
variables which are inherited from A will be reset to the values they were 
given during the original construction of the object. This is because A's 
constructor will run. In fact, every constructor above the first 
non-serialisable class constructor will also run since once the first super 
constructor is invoked (during deserialisation), it invokes its super 
constructor and so on up the inheritance tree.

Note that if you serialise a collection or an array, every element must be 
serialisable. A single non-serialisable element will cause serialisation to 
fail. Also note that while the collection interfaces are not serialisable, the 
concrete collection classes in the java api are

\subsubsection{Serialisation is Not for Statics}
Static variables are never saved as part of an object's state as they do not 
belong to the object

\subsection{Working with Dates, Numbers and Currencies}
There are a few classes needed for the exam:
\begin{itemize}
    \item \verb#java.util.Date# -- Most of the methods have been deprecated, 
    but this class can be used to bridge between the \verb#Calendar# and 
    \verb#DateFormat# classes. An instance of \verb#Date# represents and 
    mutable
    date and time, to a millisecond
    \item \verb#java.util.Calendar# -- Provides a variety of methods that help 
    to convert and manipulate dates and times
    \item \verb#java.text.DateFormat# Used to format dates using different 
    styles and locales
    \item \verb#java.text.NumberFormat# -- Used to format numbers and 
    currencies for different locales
    \item \verb#java.util.Locale# -- Used in conjunction with \verb#DateFormat# 
    and \verb#NumberFormat# to format dates, numbers and currency for specific 
    locales. If you want to manipulate dates without producing formatted 
    output, you can use the \verb#Locale# class directly with the 
    \verb#Calendar# class
\end{itemize}

\subsubsection{The Date Class}
Although most of \verb#Date#'s methods have been depricated, it's still 
acceptable to use the \verb#getTime()# and \verb#setTime()# methods, although 
it's a bit painful

\subsubsection{The Calendar Class}
The \verb#Calendar# class is designed to make date manipulation easy (compared to 
using the \verb#Date# class directly). Note that \verb#Calendar# is an abstract 
class and in order to create an instance of \verb#Calendar#, you use one of the 
overloaded \verb#getInstance()# static factory methods:
\verb#Calendar c = Calendar.getInstance()#. \verb#Calendar#'s \verb#add()# 
method lets you add or subtract units of time appropriate for whichever 
\verb#Calendar# field you specify, for example:
\begin{verbatim}
    c.add(Calendar.HOUR, -4);
    c.add(Calendar.YEAR, 2);
    c.add(Calendar.DAY_OF_WEEK, -2);
\end{verbatim}
Another important method is \verb#roll()#, which acts like \verb#add()#, except 
that when a part of a \verb#Date# is incremented and decremented, larger parts 
of the \verb#Date# are not incremented or decremented

\subsubsection{The DateFormat Class}
\verb#DateFormat# is an abstract class and has two factory methods -- 
\verb#getInstance()# and \verb#getDateInstance()#. Note that 
\verb#getDateInstance()# is overloaded to accept locales in addition to static 
fields which customise the \verb#DateFormat# instance. Another important method 
is \verb#parse()# which takes a \verb#String# formatted in the style of the 
\verb#DateFormat# instance being used, and converts the \verb#String# into a 
\verb#Date# object. Note that \verb#parse()# can throw a \verb#ParseException#

\subsubsection{The Locale Class}
The two \verb#Locale# constructors needed for the exam are:
\begin{verbatim}
    Locale(String language)
    Locale(String language, String country)
\end{verbatim}
They can be used as follows:
\begin{verbatim}
    Calendar c = Calendar.getInstance();
    Date d = c.getTime();
    Locale locIT = new Locale("it", "IT");  // Italy
    DateFormat dfUsFull = DateFormat.getDateInstance(DateFormat.FULL);
    System.out.println("US full " + dfUsFull.format(d));
    DateFormat dfItFull = DateFormat.getDateInstance(DateFormat.FULL, locIT);
    System.out.println("Italy full " + dfItFull.format(d));
\end{verbatim}
There are two other \verb#Local# methods which are needed for the exam -- 
\verb#getDisplayCountry()# and \verb#getDisplayLanguage()#. These methods allow 
you to create Strings that represent a given locale's country and language in 
terms of both the default locale and any other locale:
\begin{verbatim}
    Calendar c = Calendar.getInstance();
    Date d = c.getTime();
    Locale locBR = new Locale("pt", "BR");  // Brazil
    System.out.println("default " + locBR.getDisplayCountry());
    System.out.println("locale " + locBR.getDisplayCountry(locBR));
    System.out.println("default " + locBR.getDisplayLanguage());
    System.out.println("locale " + locBR.getDisplayLanguage(locBR));
\end{verbatim}

\subsubsection{DateFormat and NumberFormat Locales}
Note that \verb#DateFormat# and \verb#NumberFormat# objects can have their 
locales set only during instantiation -- no methods exist to change the locale 
of an existing instance

\subsubsection{The NumberFormat Class}
Like the \verb#DateFormat# class, \verb#NumberFormat# is abstract, so you'll 
typically use some version of either \verb#getInstance()# or 
\verb#getCurrencyInstance()# to create a \verb#NumberFormat# object. You use 
this class to format numbers or currency values using the \verb#format()# 
method. There are methods other than \verb#format# on the exam including, 
\verb#getMaximumFractionDigits()#, \verb#setMaximumFractionDigits#, 
\verb#parse()# and \verb#setParseIntegerOnly()#. Note that \verb#format()# 
rounds the value being displayed to the number of digits specified by 
\verb#setMaximumFractionDigits()#

\subsubsection{Instance Creation for Key java.text and java.util Classes}
\begin{center}
\begin{tabular}{ll}
    \textbf{Class} & \textbf{Key Instance Creation Options} \\
    \hline
    \verb#java.util.Date# & \verb#new Date()# \\
    & \verb#new Date(long millisecondsSince_01_01_1970)# \\
    \hline
    \verb#java.util.Calendar# & \verb#Calendar.getInstance()# \\
    & \verb#Calendar.getInstance(Locale)# \\
    \hline
    \verb#java.util.Locale# & \verb#Locale.getDefault()# \\
    & \verb#new Locale(String language)# \\
    & \verb#new Locale(String language, String country)# \\
    \hline
    \verb#java.text.DateFormat# & \verb#DateFormat.getInstance()# \\
    & \verb#DateFormat.getDateInstance()# \\
    & \verb#DateFormat.getDateInstance(style)# \\
    & \verb#DateFormat.getDateInstance(style, Locale)# \\
    \hline
    \verb#java.text.NumberFormat# & \verb#NumberFormat.getInstance()# \\
    & \verb#NumberFormat.getInstance(Locale)# \\
    & \verb#NumberFormat.getNumberInstance()# \\
    & \verb#NumberFormat.getNumberInstance(Locale)# \\
    & \verb#NumberFormat.getCurrencyInstance()# \\
    & \verb#NumberFormat.getCurrencyInstance(Locale)# \\
\end{tabular}
\end{center}

\subsection{Parsing, Tokenising and Formatting}
\subsubsection{Simple Searches}
\begin{verbatim}
    import java.util.regex.*;
    class SimpleRegex {
        public static void main(String[] args) {
            Pattern p = Pattern.compile("aba");
            Matcher m = p.matcher("abababa");
            while(m.find())
                System.out.print(m.start() + " ");
        }
    }
\end{verbatim}
This produces \verb#0 4#. In general, a regex search runs from left to right, 
and once a character has been used in a match, it cannot be reused

\subsubsection{Searches Using Metacharacters}
Regex provides a rich set of metacharacters. Those needed for the exam are:
\begin{itemize}
    \item \verb#\d# -- A digit
    \item \verb#\s# -- A whitespace character
    \item \verb#\w# -- A `word' character (letter, digit or underscore)
\end{itemize}
You can also specify sets of characters to search for using square brackets and 
ranges of characters to search for using square brackets and a dash:
\begin{itemize}
    \item \verb#[abc]# -- Searches only for a's, b's or c's
    \item \verb#[a-f]# -- Searches only for a, b, c, d, e or f characters
    \item \verb#[a-fA-F]# -- Searches for characters in the ranges a-f or A-F
\end{itemize}
Quantifiers can also be used (optionally with parentheses):
\begin{itemize}
    \item \verb#?# -- Zero or one occurrence
    \item \verb#*# -- Zero or more occurrences
    \item \verb#+# -- One or more occurrences
\end{itemize}
Negation of a group of characters can be achieved by using \verb#^# within 
square brackets

\subsubsection{The Predefined Dot}
The \verb#.# metacharacter can be used to refer to any single character

\subsubsection{Greedy Quantifiers}
When using the \verb#*#, \verb#+# and \verb#?# quantifiers, they can be fine 
tuned to produce behaviour known as `greedy', `relucant', or `possessive'. You 
only need to know `greedy' for the exam:
\begin{itemize}
    \item \verb#?# is greedy, \verb#??# is reluctant for zero or one
    \item \verb#*# is greedy, \verb#*?# is reluctant for zero or more
    \item \verb#+# is greedy, \verb#+?# is reluctant for one or more
\end{itemize}
The greedy quantifier reads as much input as possible and then works backwards 
until is finds the rightmost match. At that point, it includes everything from 
earlier in the data up to and including the char that is part of the rightmost 
match

\subsubsection{When Metacharacters and Strings Collide}
Metacharacters and Strings don't mix well -- for example,
\verb#String pattern = "\d"# results in a compiler error and the workaround is 
to escape the metacharacter with another \verb#\#. If you want to have a 
literal \verb#.# in a String which is used to compile a pattern, it must be 
double escaped (i.e. \verb#String p = "\\."#)

\subsubsection{Locating Data via Pattern Matching}
The \verb#java.util.regex.Pattern# and \verb#java.util.regex.Matcher# classes 
are used to perform pattern matching. The \verb#Pattern# class is used to hold 
a representation of a regex expression, that can be used and reused by 
instances of the \verb#Matcher# class. The \verb#Matcher# class is used to 
invoke the regex engine with the intention of performing match operations. Note 
that we create a \verb#Pattern# using the static \verb#compile(String pattern)# 
method. To create a \verb#Matcher#, we use the
\verb#Pattern.matcher(String sourceData)# method.

The \verb#Matcher.find()# method returns \verb#true# if a match is found and 
remembers the starting index of the match. The \verb#Matcher.start()# method 
can be used to obtain the starting 0-index of the match, and the 
\verb#Matcher.group()# mathod can be used to obtain the whole match.

To provide the most flexibility, \verb#Matcher.find()#, when coupled with the 
greedy quantifiers \verb#?# or \verb#*#, allow for the idea of a zero-length 
match. Zero-length matches can occur in several places:
\begin{itemize}
    \item After the last character of source data
    \item In between characters after a match has been found
    \item At the beginning of source data
    \item At the beginning of zero-length source data
\end{itemize}

\subsubsection{Searching Using the Scanner Class}
Although the \verb#java.util.Scanner# class is primarily intended for 
tokenising data, it can also be used to perform searches like the 
\verb#Pattern# and \verb#Matcher# classes. While \verb#Scanner# does not 
provide location information or search and replace functionality, it can be 
used to apply regex expressions to source data to determine how many instances 
of an expression exist in a given piece of source data. A \verb#Scanner# is 
created using \verb#Scanner s = new Scanner(System.in)# and the 
\verb#findInLine(String regex)# method returns the matching String or 
\verb#null#

\subsubsection{Tokenising with String.split()}
\verb#String#'s \verb#split(String regex)# method returns a String array 
populated with the tokens produced by the tokenising process. This is a handy 
way to tokenise relatively small pieces of data. One drawback of the 
\verb#String.split()# method is that it is not possible to inspect the tokens 
as they are produced and break out of the tokenising process early. The 
\verb#Scanner# class provides a rich API for doing such on-the-fly tokenisation 
operations

\subsubsection{Tokenising with the Scanner Class}
In addition to the basic tokenising capabilities provided by 
\verb#String.split()#, the \verb#Scanner# class offers the following features:
\begin{itemize}
    \item \verb#Scanner#s can be constructed using files, streams or Strings as 
    a source
    \item Tokenising is performed within a loop so you can exit the tokenising 
    process at any point
    \item Tokens can be converted to their appropriate primitive types 
    automatically
\end{itemize}
Note that \verb#Scanner#'s default delimiter is whitespace and that the methods 
named \verb#hasNextXxx()# test the value of the next token but do not actually 
get the token, nor do they move to the next token in the source data. The 
\verb#nextXxx()# methods all perform two functions: they get the next token and 
then they move to the next token. The \verb#Scanner# class has 
\verb#hasNextXxx()# and \verb#nextXxx()# methods for every primitive type 
except \verb#char#. It also has a \verb#useDelimiter()# method that allows you 
to set the delimiter to be any valid regex expression

\subsection{Formatting with printf() and format()}
The \verb#format()# and \verb#printf()# methods were added to 
\verb#java.io.PrintStream# in java 5 and the two methods behave exactly the 
same way. Behind the scenes, the \verb#format()# method uses the 
\verb#java.util.Formatter# class to perform the formatting work. You can use 
the \verb#Formatter# class directly if you choose, but for the exam you'll only 
have to know the basic syntax of the arguments passed to the \verb#printf()# 
and \verb#format()# methods. The basic format is:
\verb#printf("format string", argument(s))#. The format string can contain both 
normal String literal information that isn't associated with any arguments, and 
argument specific formatting data. The clue to determining whether you're 
looking at formatting data, is that the formatting data will always start with 
a percentage sign. The format strings have the following format:
\begin{verbatim}
    %[arg_index$][flags][width][.precision]conversion_char
\end{verbatim}
Note that the only required elements are the percentage sign and the conversion 
character. The format string elements needed for the exam are:
\begin{itemize}
    \item \verb#arg_index# -- An integer followed directly by a \verb#$#. This 
    indicates which argument should be printed in this position using 1-index
    \item \verb#flags# -- While many flags are available, for the exam, you'll 
    need to know:
        \begin{itemize}
            \item \verb#"-"# -- Left justify this argument
            \item \verb#"+"# -- Include a sign (\verb#+# or \verb#-#) with this 
            argument
            \item \verb#"0"# -- Pad this argument with zeros
            \item \verb#","# -- Use locale-specific grouping separators
            \item \verb#"("# -- Enclose negative numbers in parentheses
        \end{itemize}
    \item \verb#width# -- This value indicates the minimum number of characters 
    to print
    \item \verb#precision# -- For the exam, you'll only need this when 
    formatting a floating-point number, and in the case of floating point 
    numbers, precision indicates the number of digits to print after the 
    decimal point
    \item \verb#conversion# -- The type of argument you'll be formatting.  
    You'll need to know:
        \begin{itemize}
            \item \verb#b# -- boolean
            \item \verb#c# -- char
            \item \verb#d# -- integer
            \item \verb#f# -- floating point
            \item \verb#s# -- String
        \end{itemize}
\end{itemize}
Note that if there is a mismatch between the type specified in the conversion 
character and the argument, a \verb#java.util.IlligalFormatConversionException# 
will be thrown at runtime
